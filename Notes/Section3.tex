%%%%%%%%%%%%%%%%%%%%%%%%%%%%%%%%%%%%%%%%%%%%%%%%%%%%%%%%%%%%%%%%%%%%%%%%%%%%%%%%%%%%%%%%%%%%%%%%%%%%%%%%%%%%%%%%%%%%%%%%%%%%%%%%%%%%%%%%%%%%%%%%%%%%%%%%%%%%%%%%%%%
% Written By Michael Brodskiy
% Class: Embedded Design: Enabling Robotics
% Professor: S. Shazli
%%%%%%%%%%%%%%%%%%%%%%%%%%%%%%%%%%%%%%%%%%%%%%%%%%%%%%%%%%%%%%%%%%%%%%%%%%%%%%%%%%%%%%%%%%%%%%%%%%%%%%%%%%%%%%%%%%%%%%%%%%%%%%%%%%%%%%%%%%%%%%%%%%%%%%%%%%%%%%%%%%%

\documentclass[12pt]{article} 
\usepackage{alphalph}
\usepackage[utf8]{inputenc}
\usepackage[russian,english]{babel}
\usepackage{titling}
\usepackage{amsmath}
\usepackage{graphicx}
\usepackage{enumitem}
\usepackage{amssymb}
\usepackage[super]{nth}
\usepackage{everysel}
\usepackage{ragged2e}
\usepackage{geometry}
\usepackage{multicol}
\usepackage{fancyhdr}
\usepackage{cancel}
\usepackage{siunitx}
\usepackage{physics}
\usepackage{lastpage}
\usepackage{tikz}
\usepackage{mathdots}
\usepackage{yhmath}
\usepackage{cancel}
\usepackage{color}
\usepackage{array}
\usepackage{multirow}
\usepackage{gensymb}
\usepackage{tabularx}
\usepackage{extarrows}
\usepackage{booktabs}
\usetikzlibrary{fadings}
\usetikzlibrary{patterns}
\usetikzlibrary{shadows.blur}
\usetikzlibrary{shapes}

\geometry{top=1.0in,bottom=1.0in,left=1.0in,right=1.0in}
\newcommand{\subtitle}[1]{%
  \posttitle{%
    \par\end{center}
    \begin{center}\large#1\end{center}
    \vskip0.5em}%

}
\usepackage{hyperref}
\hypersetup{
colorlinks=true,
linkcolor=blue,
filecolor=magenta,      
urlcolor=blue,
citecolor=blue,
}

\pagestyle{fancy}

\lfoot[\vspace{-15pt} \hline]{\vspace{-15pt} \hline}
\rfoot[\vspace{-15pt} \hline]{\vspace{-15pt} \hline}
\cfoot[\thepage]{\thepage}
\chead[\textsc{Embedded Systems}]{\textsc{Embedded Systems}}
\lhead[\textsc{EECE2160, CRN: 32014}]{\textsc{EECE2160, CRN: 32014}}
\rhead[\textsc{Page \thepage \hspace{1pt} of \pageref{LastPage}}]{\textsc{Page \thepage \hspace{1pt} of \pageref{LastPage}}}



\pagestyle{fancy}

\title{Digital Logic Circuits}
\date{\today}
\author{Michael Brodskiy\\ \small Professor: S. Shazli}

\begin{document}

\maketitle

\thispagestyle{fancy}

\newpage

\begin{itemize}

  \item About a dozen logical operations

    \begin{itemize}

      \item Similar to algebraic operators (+, *, -, /)

    \end{itemize}

  \item In the following examples:

    \begin{itemize}

      \item $p$ = ``Today is Friday''

      \item $q$ = ``Today is my birthday''

    \end{itemize}

  \item A not operation switches (negates the truth value)

  \item Symbol: $\neg$, $\sim$, `

  \item In C and C++ the operand is !

  \item Ex. $\neg p$ = ``Today is not Friday''

  \item $\neg p=p'$

  \item An and operation is true if both operands are true

  \item Symbol: $\wedge$, $\bullet$

    \begin{itemize}

      \item It's like the ``A'' in And

    \end{itemize}

  \item In C and C++, the operand is \&\&

  \item $p\wedge q$ = ``Today is Friday and today is my birthday''

  \item $A\wedge B=A\bullet B=AB$

    \begin{center}
      \begin{tabular}[h]{|c|c|c|}
        \hline
        $p$ & $q$ & $p\wedge q$\\
        \hline
        T & T & T\\
        \hline
        T & F & F\\
        \hline
        F & T & F\\
        \hline
        F & F & F\\
        \hline
      \end{tabular}
    \end{center}

  \item An or operation is true if either operand is true

  \item Symbol: $\vee,+$

  \item In C and C++, the operand is \parallel 

  \item $p\vee q=$ ``Today is Friday or today is my birthday (or possible both)''

    \begin{center}
      \begin{tabular}[h]{|c|c|c|}
        \hline
        $p$ & $q$ & $p\vee q$\\
        \hline
        T & T & T\\
        \hline
        T & F & T\\
        \hline
        F & T & T\\
        \hline
        F & F & F\\
        \hline
      \end{tabular}
    \end{center}

  \item An exclusive or operation is true if one of the operands are true, but false if both are true

  \item Symbol: $\oplus$

  \item Often called XOR

  \item $p\oplus q=(p\vee q)\wedge\neg(p\wedge q)$

  \item $p\oplus q=$ ``Today is Friday or today is my birthday, but not both''

    \begin{center}
      \begin{tabular}[h]{|c|c|c|}
        \hline
        $p$ & $q$ & $p\oplus q$\\
        \hline
        T & T & F\\
        \hline
        T & F & T\\
        \hline
        F & T & T\\
        \hline
        F & F & F\\
        \hline
      \end{tabular}
    \end{center}

  \item Logical Operator Summary Table:

    \begin{center}
      \begin{tabular}[h]{|c|c|c|c|c|c|c|c|c|}
        \hline
        & & not & not & and & or & xor & nand & nor\\
        \hline
        $p$ & $q$ & $\neg p$ & $\neg q$ & $p\wedge q$ & $p\vee q$ & $p\oplus q$ & $p|q$ & $p\downarrow q$\\
        \hline
        T & T & F & F & T & T & F & F & F\\
        \hline
        T & F & F & T & F & T & T & T & F\\
        \hline
        F & T & T & F & F & T & T & T & F\\
        \hline
        F & F & T & T & F & F & F & T & T\\
        \hline
      \end{tabular}
    \end{center}

  \item Precedence Order (from highest to lowest):

    $$\neg,\wedge,\vee,\rightarrow,\leftrightarrow$$

  \item Not is always performed before any other operation

  \item Tautology is a statement that is always true:

    $$p\vee\neg p\text{ will always be true}$$
    $$p\wedge\neg p\text{ will always be false}$$

  \item $p\wedge T\equiv p$ — Identity Law

    \begin{center}
      \begin{tabular}[h]{|c|c|c|}
        \hline
        $p$ & $T$ & $p\wedge T$\\
        \hline
        T & T & T\\
        \hline
        F & T & F\\
        \hline
      \end{tabular}
    \end{center}

  \item $p\wedge F\equiv F$ — Domination Law

    \begin{center}
      \begin{tabular}[h]{|c|c|c|}
        \hline
        $p$ & $F$ & $p\wedge F$\\
        \hline
        T & F & T\\
        \hline
        F & F & F\\
        \hline
      \end{tabular}
    \end{center}

  \item $p\wedge p\equiv p$ — Idempotent Law

    \begin{center}
      \begin{tabular}[h]{|c|c|c|}
        \hline
        $p$ & $p$ & $p\wedge p$\\
        \hline
        T & T & T\\
        \hline
        F & F & F\\
        \hline
      \end{tabular}
    \end{center}

  \item $p\wedge q\equiv q\wedge p$ — Commutative Law

    \begin{center}
      \begin{tabular}[h]{|c|c|c|c|}
        \hline
        $p$ & $q$ & $p\wedge q$ & $q\wedge p$\\
        \hline
        T & T & T & T\\
        \hline
        T & F & F & F\\
        \hline
        F & T & F & F\\
        \hline
        F & F & F & F\\
        \hline
      \end{tabular}
    \end{center}

  \item $(p\wedge q)\wedge r\equiv p\wedge (q\wedge r)$ — Associative Law

    \begin{center}
      \begin{tabular}[h]{|c|c|c|c|c|c|c|}
        \hline
        $p$ & $q$ & $r$ & $p\wedge q$ & $(p\wedge q)\wedge r$ & $q\wedge r$ & $p\wedge(q\wedge r)$\\
        \hline
        T & T & T & T & T & T & T\\
        \hline
        T & T & F & T & F & F & F\\
        \hline
        T & F & T & F & F & F & F\\
        \hline
        T & T & F & F & F & F & F\\
        \hline
        T & F & F & F & F & F & F\\
        \hline
        F & T & T & F & F & T & F\\
        \hline
        F & T & F & F & F & F & F\\
        \hline
        F & F & T & F & F & F & F\\
        \hline
        F & F & F & F & F & F & F\\
        \hline
      \end{tabular}
    \end{center}

  \item $p\vee T\equiv T$ — Identity Law

  \item $p\vee F\equiv p$ — Domination Law

  \item $p\vee p\equiv p$ — Idempotent Law

  \item $p\vee q \equiv q\vee p$ — Commutative Law

  \item $(p\vee q)\vee r\equiv p\vee(q\vee r)$ — Associative Law

\end{itemize}

\end{itemize}

\end{document}


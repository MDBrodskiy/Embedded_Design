%%%%%%%%%%%%%%%%%%%%%%%%%%%%%%%%%%%%%%%%%%%%%%%%%%%%%%%%%%%%%%%%%%%%%%%%%%%%%%%%%%%%%%%%%%%%%%%%%%%%%%%%%%%%%%%%%%%%%%%%%%%%%%%%%%%%%%%%%%%%%%%%%%%%%%%%%%%%%%%%%%%
% Written By Michael Brodskiy
% Class: Embedded Design: Enabling Robotics
% Professor: S. Shazli
%%%%%%%%%%%%%%%%%%%%%%%%%%%%%%%%%%%%%%%%%%%%%%%%%%%%%%%%%%%%%%%%%%%%%%%%%%%%%%%%%%%%%%%%%%%%%%%%%%%%%%%%%%%%%%%%%%%%%%%%%%%%%%%%%%%%%%%%%%%%%%%%%%%%%%%%%%%%%%%%%%%

\include{Includes.tex}

\pagestyle{fancy}

\title{Digital Logic Circuits}
\date{\today}
\author{Michael Brodskiy\\ \small Professor: S. Shazli}

\begin{document}

\maketitle

\thispagestyle{fancy}

\newpage

\begin{itemize}

  \item Canonical POS Forms

    \begin{itemize}

      \item Product of Sums / Conjunctive Normal Form / Maxterm Expansion

        \begin{center}
          \begin{tabular}[h!]{c c c | l}
            \textbf{A} & \textbf{B} & \textbf{C} & Maxterms\\
            \hline
             0 & 0 & 0 & $\bold{A} + \bold{B} + \bold{C}= M0$\\
             0 & 0 & 1 & $\bold{A} + \bold{B} + \bold{\bar{C}}= M1$\\
             0 & 1 & 0 & $\bold{A} + \bold{\bar{B}} + \bold{C}= M2$\\
             0 & 1 & 1 & $\bold{A} + \bold{\bar{B}} + \bold{\bar{C}}= M3$\\
             1 & 0 & 0 & $\bold{\bar{A}} + \bold{B} + \bold{C}= M4$\\
             1 & 0 & 1 & $\bold{\bar{A}} + \bold{B} + \bold{C}= M5$\\
             1 & 1 & 0 & $\bold{\bar{A}} + \bold{\bar{B}} + \bold{C}= M6$\\
             1 & 1 & 1 & $\bold{\bar{A}} + \bold{\bar{B}} + \bold{\bar{C}}= M7$\\
             \hline
          \end{tabular}
        \end{center}

      \item This can be rewritten as:

        $$\bold{F}=(\bold{A}+\bold{B}+\bold{C})(\bold{A}+\bold{B}+\bold{\bar{C}})(\bold{A}+\bold{\bar{B}}+\bold{C})=\prod M(0,1,2)$$

      \item The terms numbered $M<\#>$ is known as the maxterm shorthand

      \item It is best to choose whichever form (product or sum) gives the least terms

    \end{itemize}

  \item Sample Design (or Quiz) Problem

    \begin{itemize}

      \item A sound alarm is connected to three environmental sensors. The alarm is triggered only when any two out of three, or all three inputs from the sensors are high

        \begin{enumerate}

          \item Set up a truth table for the output of the alarm system

            \vspace{10pt}

            \begin{center}
              \begin{tabular}[h!]{c c c | l}
                a & b & c & f(a,b,c)\\
                \hline
                0 & 0 & 0 & 0\\
                0 & 0 & 1 & 0\\
                0 & 1 & 0 & 0\\
                0 & 1 & 1 & 1\\
                1 & 0 & 0 & 0\\
                1 & 0 & 1 & 1\\
                1 & 1 & 0 & 1\\
                1 & 1 & 1 & 1\\
                \hline
              \end{tabular}
            \end{center}

            \vspace{10pt}

          \item Find the boolean expression for the output

            \begin{itemize}

              \item As a sum of products

                $$\bar{a}bc+a\bar{b}c+ab\bar{c}+abc$$

            \end{itemize}

          \item Draw the combinational logic circuit diagram for output

            The number of ``and'' gates for any sum of products would be proportional to the amount of terms. For the above from (2), there will be 4 ``and'' gates leading to an ``or'' gate

        \end{enumerate}

    \end{itemize}

    \begin{itemize}

      \item The first step would be to create a ``black box'' and define and name inputs and outputs

        \begin{itemize}

          \item Try to give meaningful names

          \item Not just something like $i,j,k$, etc.

        \end{itemize}

    \end{itemize}

\end{itemize}

\end{document}


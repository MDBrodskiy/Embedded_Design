%%%%%%%%%%%%%%%%%%%%%%%%%%%%%%%%%%%%%%%%%%%%%%%%%%%%%%%%%%%%%%%%%%%%%%%%%%%%%%%%%%%%%%%%%%%%%%%%%%%%%%%%%%%%%%%%%%%%%%%%%%%%%%%%%%%%%%%%%%%%%%%%%%%%%%%%%%%%%%%%%%%
% Written By Michael Brodskiy
% Class: Embedded Design: Enabling Robotics
% Professor: S. Shazli
%%%%%%%%%%%%%%%%%%%%%%%%%%%%%%%%%%%%%%%%%%%%%%%%%%%%%%%%%%%%%%%%%%%%%%%%%%%%%%%%%%%%%%%%%%%%%%%%%%%%%%%%%%%%%%%%%%%%%%%%%%%%%%%%%%%%%%%%%%%%%%%%%%%%%%%%%%%%%%%%%%%

\documentclass[12pt]{article} 
\usepackage{alphalph}
\usepackage[utf8]{inputenc}
\usepackage[russian,english]{babel}
\usepackage{titling}
\usepackage{amsmath}
\usepackage{graphicx}
\usepackage{enumitem}
\usepackage{amssymb}
\usepackage[super]{nth}
\usepackage{everysel}
\usepackage{ragged2e}
\usepackage{geometry}
\usepackage{multicol}
\usepackage{fancyhdr}
\usepackage{cancel}
\usepackage{siunitx}
\usepackage{physics}
\usepackage{lastpage}
\usepackage{tikz}
\usepackage{mathdots}
\usepackage{yhmath}
\usepackage{cancel}
\usepackage{color}
\usepackage{array}
\usepackage{multirow}
\usepackage{gensymb}
\usepackage{tabularx}
\usepackage{extarrows}
\usepackage{booktabs}
\usetikzlibrary{fadings}
\usetikzlibrary{patterns}
\usetikzlibrary{shadows.blur}
\usetikzlibrary{shapes}

\geometry{top=1.0in,bottom=1.0in,left=1.0in,right=1.0in}
\newcommand{\subtitle}[1]{%
  \posttitle{%
    \par\end{center}
    \begin{center}\large#1\end{center}
    \vskip0.5em}%

}
\usepackage{hyperref}
\hypersetup{
colorlinks=true,
linkcolor=blue,
filecolor=magenta,      
urlcolor=blue,
citecolor=blue,
}

\pagestyle{fancy}

\lfoot[\vspace{-15pt} \hline]{\vspace{-15pt} \hline}
\rfoot[\vspace{-15pt} \hline]{\vspace{-15pt} \hline}
\cfoot[\thepage]{\thepage}
\chead[\textsc{Embedded Systems}]{\textsc{Embedded Systems}}
\lhead[\textsc{EECE2160, CRN: 32014}]{\textsc{EECE2160, CRN: 32014}}
\rhead[\textsc{Page \thepage \hspace{1pt} of \pageref{LastPage}}]{\textsc{Page \thepage \hspace{1pt} of \pageref{LastPage}}}



\def\code#1{\texttt{#1}}

\pagestyle{fancy}

\title{Object Oriented C++}
\date{\today}
\author{Michael Brodskiy\\ \small Professor: S. Shazli}

\begin{document}

\maketitle

\thispagestyle{fancy}

\newpage

\begin{itemize}

  \item Object-oriented programming (OOP) is more natural to describe the interactions between ``things'' (\textit{i}.\textit{e}. objects)

  \item OOP provides better code reuse

    \begin{itemize}

      \item Commonalities among objects described by a class
        
      \item Commonalities among classes described by a base class (inheritance)
        
    \end{itemize}

  \item Objects know what to do using their attributes: Each object responds differently to ``What is your name?''

  \item OOP provides encapsulation: hide data that do not have to be visible to the other objects or protect data from unintentional, inconsistent changes

  \item Objects Definition

    \begin{itemize}

      \item An object can be a ``concrete and tangible'' entity that can be separated with unique properties

      \item An object can be abstract and does not have to be tangible

    \end{itemize}

  \item Objects' Three Properties

    \begin{itemize}

      \item The \code{Each} object is unique and can be identified (object's name) using name, serial number, relationship with another object, etc.

      \item Each object has a set of attributes (data members), such as location, speed, size, address, phone number, on/off, etc.

      \item Each object has unique behaviors (functions/methods), such as ring (phone), accelerate and move (car), take picture (camera), etc.

    \end{itemize}

  \item Class Definition — Access Control

    \begin{itemize}

      \item Information hiding — Encapsulation

        \begin{itemize}

          \item To prevent the internal representation from direct access from outside the class

        \end{itemize}

      \item Access Specifier Keywords

        \begin{itemize}

          \item \code{public}

            \begin{itemize}

              \item May be accessible from anywhere within the program

            \end{itemize}

          \item \code{private}

            \begin{itemize}

              \item May be accessed only by the member functions, and friends of this class, not open for non-member functions

            \end{itemize}

          \item \code{protected}

            \begin{itemize}

              \item Acts as public for derived classes (child)

              \item Behaves as private for the rest of the program

            \end{itemize}

        \end{itemize}

      \item Difference between classes and structs in C++

        \begin{itemize}

          \item The default access specifier is private in classes

          \item The default access specifier is public in structs

        \end{itemize}

    \end{itemize}

\end{itemize}

\end{document}


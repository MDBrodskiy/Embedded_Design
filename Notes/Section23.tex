%%%%%%%%%%%%%%%%%%%%%%%%%%%%%%%%%%%%%%%%%%%%%%%%%%%%%%%%%%%%%%%%%%%%%%%%%%%%%%%%%%%%%%%%%%%%%%%%%%%%%%%%%%%%%%%%%%%%%%%%%%%%%%%%%%%%%%%%%%%%%%%%%%%%%%%%%%%%%%%%%%%
% Written By Michael Brodskiy
% Class: Embedded Design: Enabling Robotics
% Professor: S. Shazli
%%%%%%%%%%%%%%%%%%%%%%%%%%%%%%%%%%%%%%%%%%%%%%%%%%%%%%%%%%%%%%%%%%%%%%%%%%%%%%%%%%%%%%%%%%%%%%%%%%%%%%%%%%%%%%%%%%%%%%%%%%%%%%%%%%%%%%%%%%%%%%%%%%%%%%%%%%%%%%%%%%%

\documentclass[12pt]{article} 
\usepackage{alphalph}
\usepackage[utf8]{inputenc}
\usepackage[russian,english]{babel}
\usepackage{titling}
\usepackage{amsmath}
\usepackage{graphicx}
\usepackage{enumitem}
\usepackage{amssymb}
\usepackage[super]{nth}
\usepackage{everysel}
\usepackage{ragged2e}
\usepackage{geometry}
\usepackage{multicol}
\usepackage{fancyhdr}
\usepackage{cancel}
\usepackage{siunitx}
\usepackage{physics}
\usepackage{lastpage}
\usepackage{tikz}
\usepackage{mathdots}
\usepackage{yhmath}
\usepackage{cancel}
\usepackage{color}
\usepackage{array}
\usepackage{multirow}
\usepackage{gensymb}
\usepackage{tabularx}
\usepackage{extarrows}
\usepackage{booktabs}
\usetikzlibrary{fadings}
\usetikzlibrary{patterns}
\usetikzlibrary{shadows.blur}
\usetikzlibrary{shapes}

\geometry{top=1.0in,bottom=1.0in,left=1.0in,right=1.0in}
\newcommand{\subtitle}[1]{%
  \posttitle{%
    \par\end{center}
    \begin{center}\large#1\end{center}
    \vskip0.5em}%

}
\usepackage{hyperref}
\hypersetup{
colorlinks=true,
linkcolor=blue,
filecolor=magenta,      
urlcolor=blue,
citecolor=blue,
}

\pagestyle{fancy}

\lfoot[\vspace{-15pt} \hline]{\vspace{-15pt} \hline}
\rfoot[\vspace{-15pt} \hline]{\vspace{-15pt} \hline}
\cfoot[\thepage]{\thepage}
\chead[\textsc{Embedded Systems}]{\textsc{Embedded Systems}}
\lhead[\textsc{EECE2160, CRN: 32014}]{\textsc{EECE2160, CRN: 32014}}
\rhead[\textsc{Page \thepage \hspace{1pt} of \pageref{LastPage}}]{\textsc{Page \thepage \hspace{1pt} of \pageref{LastPage}}}



\def\code#1{\texttt{#1}}

\pagestyle{fancy}

\title{The Final Lab Project}
\date{\today}
\author{Michael Brodskiy\\ \small Professor: S. Shazli}

\begin{document}

\maketitle

\thispagestyle{fancy}

\newpage

\begin{itemize}

  \item In this project, you will utilize the 7-segment displays on the DE1-SoC board to display characters, decimal, and hexadecimal values using object oriented design in C++

  \item The DE1-SoC has six 7-segment displays controlled by two parallel ports

  \item Each segment (6 to 0) is controlled by one bit, with the first parallel port controlling the first four displays (HEX3, HEX2, HEX1, \& HEX0), and the second parallel port controlling the last two displays (HEX5 \& HEX4)

  \item Data can be written into these two registers, and read back, by using word operations

  \item The seven segment display table:

    \begin{center}
      \begin{tabular}[h!]{| c | c | c | c | c | c | c | c | c | c |}
        \hline
        \# & 6 & 5 & 4 & 3 & 2 & 1 & 0 & Decimal & Hex \\
        \hline
        0 & 0 & 1 & 1 & 1 & 1 & 1 & 1 & 63 & 0x3F\\
        \hline
        1 & 0 & 0 & 0 & 0 & 1 & 1 & 0 & 6 & 0x6\\
        \hline
        2 & 1 & 0 & 1 & 1 & 0 & 1 & 1 & 91 & 0x5B\\
        \hline
        3 & 1 & 0 & 0 & 1 & 1 & 1 & 1 & 79 & 0x4F\\
        \hline
        4 & 1 & 1 & 0 & 0 & 1 & 1 & 0 & 102 & 0x66\\
        \hline
        5 & 1 & 1 & 0 & 1 & 1 & 0 & 1 & 109 & 0x6D\\
        \hline
        6 & 1 & 1 & 1 & 1 & 1 & 0 & 1 & 125 & 0x7D\\
        \hline
        7 & 0 & 0 & 0 & 0 & 1 & 1 & 1 & 7 & 0x7\\
        \hline
        8 & 1 & 1 & 1 & 1 & 1 & 1 & 1 & 127 & 0x7F\\
        \hline
        9 & 1 & 1 & 0 & 1 & 1 & 1 & 1 & 111 & 0x6F\\
        \hline
        A & 1 & 1 & 1 & 0 & 1 & 1 & 1 & 119 & 0x77\\
        \hline
        b & 1 & 1 & 1 & 1 & 1 & 0 & 0 & 124 & 0x7C\\
        \hline
        C & 0 & 1 & 1 & 1 & 0 & 0 & 1 & 57 & 0x39\\
        \hline
        d & 1 & 0 & 1 & 1 & 1 & 1 & 0 & 94 & 0x5E\\
        \hline
        e & 1 & 1 & 1 & 1 & 0 & 0 & 1 & 121 & 0x79\\
        \hline
        f & 1 & 1 & 1 & 0 & 0 & 0 & 1 & 113 & 0x71\\
        \hline
      \end{tabular}
    \end{center}

  \item The \texttt{DE1SoCfpga} class from the previous lab should be converted to a header file called \texttt{DE1SoCfpga.h}

  \item It should be implemented in a file called \texttt{DE1SoCfpga.cpp}

  \item The following functions need to be part of the header:

    \begin{itemize}

      \item Constructor initializing the memory-mapped I/O

      \item Destructor finalizing the memory-mapped I/O

      \item Function \texttt{RegisterWrite(offset, value)}, which writes a value into a register given its offset

      \item Function \texttt{RegisterRead(offset)}, returning the value read from a register given its offset

    \end{itemize}

  \item The class \texttt{DE1SoCfpga} will be the base class initializing memory and controlling register access

  \item There are two offsets for the hex displays:

    \begin{center}
      \texttt{const unsigned int HEX3\_HEX0\_Base = 0x00000020;}
    \end{center}

    \begin{center}
      and
    \end{center}

    \begin{center}
      \texttt{const unsigned int HEX5\_HEX4\_Base = 0x00000030;}
    \end{center}

  \item For the seven segment display, a new class should be created, called \texttt{SevenSegment}, with the following functionality:

    \begin{itemize}

      \item Two private data members: \texttt{unsigned int reg0\_hexValue} and \texttt{unsigned int reg1\_hexValue} representing the state of the two 7-segment display registers

      \item These variables should be updated every time a new value is written to the corresponding registers

      \item A class constructor that initializes the private data members

      \item A class destructor that calls \texttt{Hex\_ClearAll()} function to clear the displays

      \item A public function called \texttt{Hex\_ClearAll()} that clears (turns off) all the 7-segment displays

      \item A public function called \texttt{Hex\_ClearSpecific(int index)} that clears (turns off) a specified 7-segment display, specified by \texttt{index}, where the index (0 to 5) represents one of the six displays

      \item A public function called \texttt{Hex\_WriteSpecific(int display\_id,  int  value)} that writes a hexadecimal digit specified by a decimal value(0 to 15) to the specified 7-segment display, specified by \texttt{display\_id}, where the \texttt{display\_id} (0 to 5) represents one of the six displays

      \item A public function called \texttt{Hex\_WriteNumber(int number)} that  writes  a  positive  or  negative number on the 7-segment displays 

      \item Use the \texttt{main()} function provided to test the file

    \end{itemize}

  \item The class \texttt{SevenSegment} inherits the class \texttt{DE1SoCfpga} and uses the functions \texttt{RegisterWrite(...)} and \texttt{RegisterRead(...)} as needed

  \item Create a global \texttt{const} array with 16 elements (see the declaration below)

  \item Each element contains the decimal (or Hex) code from the table above that represents the display segments to be turned ON or OFF when writing to a 7-segment display

  \item The index corresponds to the digits to write on the 7-segment display

  \item Include the array in the \texttt{SevenSegment.h} file to use when writing to the 7-segment displays. The arrays shoulder have the following format:

    \begin{center}
      \texttt{const unsigned int bit\_values[16] = \{\ldots\};}
    \end{center}

  \item The seven segment displays should then be integrated with \texttt{LEDControl} from the previous lab

\end{itemize}

\end{document}


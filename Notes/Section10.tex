%%%%%%%%%%%%%%%%%%%%%%%%%%%%%%%%%%%%%%%%%%%%%%%%%%%%%%%%%%%%%%%%%%%%%%%%%%%%%%%%%%%%%%%%%%%%%%%%%%%%%%%%%%%%%%%%%%%%%%%%%%%%%%%%%%%%%%%%%%%%%%%%%%%%%%%%%%%%%%%%%%%
% Written By Michael Brodskiy
% Class: Embedded Design: Enabling Robotics
% Professor: S. Shazli
%%%%%%%%%%%%%%%%%%%%%%%%%%%%%%%%%%%%%%%%%%%%%%%%%%%%%%%%%%%%%%%%%%%%%%%%%%%%%%%%%%%%%%%%%%%%%%%%%%%%%%%%%%%%%%%%%%%%%%%%%%%%%%%%%%%%%%%%%%%%%%%%%%%%%%%%%%%%%%%%%%%

\documentclass[12pt]{article} 
\usepackage{alphalph}
\usepackage[utf8]{inputenc}
\usepackage[russian,english]{babel}
\usepackage{titling}
\usepackage{amsmath}
\usepackage{graphicx}
\usepackage{enumitem}
\usepackage{amssymb}
\usepackage[super]{nth}
\usepackage{everysel}
\usepackage{ragged2e}
\usepackage{geometry}
\usepackage{multicol}
\usepackage{fancyhdr}
\usepackage{cancel}
\usepackage{siunitx}
\usepackage{physics}
\usepackage{lastpage}
\usepackage{tikz}
\usepackage{mathdots}
\usepackage{yhmath}
\usepackage{cancel}
\usepackage{color}
\usepackage{array}
\usepackage{multirow}
\usepackage{gensymb}
\usepackage{tabularx}
\usepackage{extarrows}
\usepackage{booktabs}
\usetikzlibrary{fadings}
\usetikzlibrary{patterns}
\usetikzlibrary{shadows.blur}
\usetikzlibrary{shapes}

\geometry{top=1.0in,bottom=1.0in,left=1.0in,right=1.0in}
\newcommand{\subtitle}[1]{%
  \posttitle{%
    \par\end{center}
    \begin{center}\large#1\end{center}
    \vskip0.5em}%

}
\usepackage{hyperref}
\hypersetup{
colorlinks=true,
linkcolor=blue,
filecolor=magenta,      
urlcolor=blue,
citecolor=blue,
}

\pagestyle{fancy}

\lfoot[\vspace{-15pt} \hline]{\vspace{-15pt} \hline}
\rfoot[\vspace{-15pt} \hline]{\vspace{-15pt} \hline}
\cfoot[\thepage]{\thepage}
\chead[\textsc{Embedded Systems}]{\textsc{Embedded Systems}}
\lhead[\textsc{EECE2160, CRN: 32014}]{\textsc{EECE2160, CRN: 32014}}
\rhead[\textsc{Page \thepage \hspace{1pt} of \pageref{LastPage}}]{\textsc{Page \thepage \hspace{1pt} of \pageref{LastPage}}}



\pagestyle{fancy}

\title{K-Map Analysis}
\date{\today}
\author{Michael Brodskiy\\ \small Professor: S. Shazli}

\begin{document}

\maketitle

\thispagestyle{fancy}

\newpage

\begin{itemize}

  \item Three-Variable K-Map

    \begin{itemize}

      \item The following truth table:

        \begin{center}
        \begin{tabular}[h!]{| c c c | c |}
          \hline
          a & b & c & $f$(a,b,c)\\
          \hline
          0 & 0 & 0 & 0\\
          0 & 0 & 1 & 1\\
          0 & 1 & 0 & 0\\
          0 & 1 & 1 & 1\\
          1 & 0 & 0 & 0\\
          1 & 0 & 1 & 1\\
          1 & 1 & 0 & 0\\
          1 & 1 & 1 & 1\\
          \hline
        \end{tabular}
      \end{center}

      \item Would be converted to the following K-Map:

        \begin{center}
        \begin{tabular}[h!]{c | c c c c |}
          a \textbackslash bc & (0,0) & (0,1) & (1,1) & (1,0)\\
          \hline
          0 & 0 & 1 & 1 & 0\\
          \hline
          1 & 0 & 1 & 1 & 0\\
          \hline
        \end{tabular}
      \end{center}

    \item By grouping $2^n$ adjacent ones, the boolean expression becomes: $f$(a,b,c)$=c$

    \end{itemize}

  \item Converting to K-Maps from boolean expressions

    \begin{itemize}

      \item $f$(a,b,c) $=$ a'b'c' + ab'c' + abc' + ab'

        \begin{center}
          \begin{tabular}[h!]{c | c c c c |}
            a \textbackslash bc & (0,0) & (0,1) & (1,1) & (1,0)\\
            \hline
            0 & 1 & 0 & 0 & 0\\
            1 & 1 & 1 & 0 & 1\\
            \hline
          \end{tabular}
        \end{center}

      \item $f$(a,b,c) $=$ a'b' + a'c'+ bc' + ab + b'c

        \begin{center}
          \begin{tabular}[h!]{c | c c c c |}
            a \textbackslash bc & (0,0) & (0,1) & (1,1) & (1,0)\\
            \hline
            0 & 1 & 1 & 0 & 1\\
            1 & 0 & 0 & 1 & 1\\
            \hline
          \end{tabular}
        \end{center}

    \end{itemize}

  \item K-Maps with Don't Cares (``x'')

    \begin{itemize}

      \item The following is an example K-Map with don't cares

        \begin{center}
          \begin{tabular}[h!]{c | c c c c |}
            AB \textbackslash CD & 00 & 01 & 11 & 10\\
            \hline
            00 & 1 & 0 & x & 1\\
            \hline
            01 & 0 & x & x & 1\\
            \hline
            11 & 1 & 1 & x & x\\
            \hline
            10 & 1 & 1 & 1 & 1\\
            \hline
          \end{tabular}
        \end{center}

    \end{itemize}

  \item Don't cares can be assigned either a 0 or 1

  \item It is necessary to create groups as big as possible

  \item The above K-Map becomes the expression $f$(a,b,c) $=$ A + $\bar{\text{B}}\bar{\text{D}}$ + C

\end{itemize}

\end{document}


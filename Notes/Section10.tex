%%%%%%%%%%%%%%%%%%%%%%%%%%%%%%%%%%%%%%%%%%%%%%%%%%%%%%%%%%%%%%%%%%%%%%%%%%%%%%%%%%%%%%%%%%%%%%%%%%%%%%%%%%%%%%%%%%%%%%%%%%%%%%%%%%%%%%%%%%%%%%%%%%%%%%%%%%%%%%%%%%%
% Written By Michael Brodskiy
% Class: Embedded Design: Enabling Robotics
% Professor: S. Shazli
%%%%%%%%%%%%%%%%%%%%%%%%%%%%%%%%%%%%%%%%%%%%%%%%%%%%%%%%%%%%%%%%%%%%%%%%%%%%%%%%%%%%%%%%%%%%%%%%%%%%%%%%%%%%%%%%%%%%%%%%%%%%%%%%%%%%%%%%%%%%%%%%%%%%%%%%%%%%%%%%%%%

\include{Includes.tex}

\pagestyle{fancy}

\title{K-Map Analysis}
\date{\today}
\author{Michael Brodskiy\\ \small Professor: S. Shazli}

\begin{document}

\maketitle

\thispagestyle{fancy}

\newpage

\begin{itemize}

  \item Three-Variable K-Map

    \begin{itemize}

      \item The following truth table:

        \begin{center}
        \begin{tabular}[h!]{| c c c | c |}
          \hline
          a & b & c & $f$(a,b,c)\\
          \hline
          0 & 0 & 0 & 0\\
          0 & 0 & 1 & 1\\
          0 & 1 & 0 & 0\\
          0 & 1 & 1 & 1\\
          1 & 0 & 0 & 0\\
          1 & 0 & 1 & 1\\
          1 & 1 & 0 & 0\\
          1 & 1 & 1 & 1\\
          \hline
        \end{tabular}
      \end{center}

      \item Would be converted to the following K-Map:

        \begin{center}
        \begin{tabular}[h!]{c | c c c c |}
          a \textbackslash bc & (0,0) & (0,1) & (1,1) & (1,0)\\
          \hline
          0 & 0 & 1 & 1 & 0\\
          \hline
          1 & 0 & 1 & 1 & 0\\
          \hline
        \end{tabular}
      \end{center}

    \item By grouping $2^n$ adjacent ones, the boolean expression becomes: $f$(a,b,c)$=c$

    \end{itemize}

  \item Converting to K-Maps from boolean expressions

    \begin{itemize}

      \item $f$(a,b,c) $=$ a'b'c' + ab'c' + abc' + ab'

        \begin{center}
          \begin{tabular}[h!]{c | c c c c |}
            a \textbackslash bc & (0,0) & (0,1) & (1,1) & (1,0)\\
            \hline
            0 & 1 & 0 & 0 & 0\\
            1 & 1 & 1 & 0 & 1\\
            \hline
          \end{tabular}
        \end{center}

      \item $f$(a,b,c) $=$ a'b' + a'c'+ bc' + ab + b'c

        \begin{center}
          \begin{tabular}[h!]{c | c c c c |}
            a \textbackslash bc & (0,0) & (0,1) & (1,1) & (1,0)\\
            \hline
            0 & 1 & 1 & 0 & 1\\
            1 & 0 & 0 & 1 & 1\\
            \hline
          \end{tabular}
        \end{center}

    \end{itemize}

  \item K-Maps with Don't Cares (``x'')

    \begin{itemize}

      \item The following is an example K-Map with don't cares

        \begin{center}
          \begin{tabular}[h!]{c | c c c c |}
            AB \textbackslash CD & 00 & 01 & 11 & 10\\
            \hline
            00 & 1 & 0 & x & 1\\
            \hline
            01 & 0 & x & x & 1\\
            \hline
            11 & 1 & 1 & x & x\\
            \hline
            10 & 1 & 1 & 1 & 1\\
            \hline
          \end{tabular}
        \end{center}

    \end{itemize}

  \item Don't cares can be assigned either a 0 or 1

  \item It is necessary to create groups as big as possible

  \item The above K-Map becomes the expression $f$(a,b,c) $=$ A + $\bar{\text{B}}\bar{\text{D}}$ + C

\end{itemize}

\end{document}


%%%%%%%%%%%%%%%%%%%%%%%%%%%%%%%%%%%%%%%%%%%%%%%%%%%%%%%%%%%%%%%%%%%%%%%%%%%%%%%%%%%%%%%%%%%%%%%%%%%%%%%%%%%%%%%%%%%%%%%%%%%%%%%%%%%%%%%%%%%%%%%%%%%%%%%%%%%%%%%%%%%
% Written By Michael Brodskiy
% Class: Embedded Design: Enabling Robotics
% Professor: S. Shazli
%%%%%%%%%%%%%%%%%%%%%%%%%%%%%%%%%%%%%%%%%%%%%%%%%%%%%%%%%%%%%%%%%%%%%%%%%%%%%%%%%%%%%%%%%%%%%%%%%%%%%%%%%%%%%%%%%%%%%%%%%%%%%%%%%%%%%%%%%%%%%%%%%%%%%%%%%%%%%%%%%%%

\documentclass[12pt]{article} 
\usepackage{alphalph}
\usepackage[utf8]{inputenc}
\usepackage[russian,english]{babel}
\usepackage{titling}
\usepackage{amsmath}
\usepackage{graphicx}
\usepackage{enumitem}
\usepackage{amssymb}
\usepackage[super]{nth}
\usepackage{everysel}
\usepackage{ragged2e}
\usepackage{geometry}
\usepackage{multicol}
\usepackage{fancyhdr}
\usepackage{cancel}
\usepackage{siunitx}
\usepackage{physics}
\usepackage{lastpage}
\usepackage{tikz}
\usepackage{mathdots}
\usepackage{yhmath}
\usepackage{cancel}
\usepackage{color}
\usepackage{array}
\usepackage{multirow}
\usepackage{gensymb}
\usepackage{tabularx}
\usepackage{extarrows}
\usepackage{booktabs}
\usetikzlibrary{fadings}
\usetikzlibrary{patterns}
\usetikzlibrary{shadows.blur}
\usetikzlibrary{shapes}

\geometry{top=1.0in,bottom=1.0in,left=1.0in,right=1.0in}
\newcommand{\subtitle}[1]{%
  \posttitle{%
    \par\end{center}
    \begin{center}\large#1\end{center}
    \vskip0.5em}%

}
\usepackage{hyperref}
\hypersetup{
colorlinks=true,
linkcolor=blue,
filecolor=magenta,      
urlcolor=blue,
citecolor=blue,
}

\pagestyle{fancy}

\lfoot[\vspace{-15pt} \hline]{\vspace{-15pt} \hline}
\rfoot[\vspace{-15pt} \hline]{\vspace{-15pt} \hline}
\cfoot[\thepage]{\thepage}
\chead[\textsc{Embedded Systems}]{\textsc{Embedded Systems}}
\lhead[\textsc{EECE2160, CRN: 32014}]{\textsc{EECE2160, CRN: 32014}}
\rhead[\textsc{Page \thepage \hspace{1pt} of \pageref{LastPage}}]{\textsc{Page \thepage \hspace{1pt} of \pageref{LastPage}}}



\pagestyle{fancy}

\title{Digital Logic Minimization}
\date{\today}
\author{Michael Brodskiy\\ \small Professor: S. Shazli}

\begin{document}

\maketitle

\thispagestyle{fancy}

\newpage

\begin{itemize}

  \item Definitions

    \begin{itemize}

      \item Literal: $x_i, x_i'$

      \item Product Term: $x_2x_1'x_0$

      \item Sum Term: $x_2 + x_1' + x_0$

    \end{itemize}

  \item Minterm of $n$ variables

    \begin{itemize}

      \item A product of $n$ literals in which every variable appears exactly once

      \item  $f(a,b,c,d)$: $ab'cd'$, $a'bc'd'$

    \end{itemize}

  \item Maxterm of $n$ variables

    \begin{itemize}

      \item A sum of $n$ literals in which every variable appears exactly once

      \item $f(a,b,c,d)$: $a'+b+c+d$, $a'+b'+c+d$

    \end{itemize}

  \item Input:

    \begin{itemize}
        
      \item Boolean expression of $n$ binary variables

    \end{itemize}

  \item Goal

    \begin{itemize}

      \item Simplification of the expression

      \item We want to minimize \# terms and \# literals

    \end{itemize}

  \item Applications

    \begin{itemize}

      \item Logic: rule reduction

      \item Hardware Design: cost and performance optimization

      \item Cost (wires, gates): \# literals, product terms, sum terms

      \item Performance: speed, reliability

    \end{itemize}

  \item Boolean Optimization

    \begin{itemize}

      \item Function can be represented by sum of minterms:

        $$f(A,B)=A'B+AB'+AD$$

      \item This is not minimal

      \item We want to minimize the number of literals and terms

      \item We can factor out common terms

        $$A’B+AB’+AB= A’B+AB’+AB+AB$$
        $$=(A’+A)B+A(B’+B)=B+A$$

      \item Thus, we have $f(A,B)=A + B$
        
    \end{itemize}

  \item Karnaugh Maps (K-Maps)

    \begin{itemize}

      \item Developed by Maurice Karnaugh

      \item Essentially a truth table in 2 dimensions

      \item For $f(A,B)=A+B$

        \begin{center}
          \begin{tabular}[h!]{| l | c | c | c | c |}
            \hline
            ID & $A$ & $B$ & $f$ & \\
            \hline
            0 & 0 & 0 & 0 & \\
            \hline
            1 & 0 & 1 & 1 & $A'B$\\
            \hline
            2 & 1 & 0 & 1 & $AB'$\\
            \hline
            3 & 1 & 1 & 1 & $AB$\\
            \hline
          \end{tabular}
        \end{center}

      \item Is converted to

      \begin{center}
        \begin{tabular}[h!]{c | c | c}
          & $B=0$ & $B=1$\\
          \hline
          $A=0$ & 0 & 1 \\
          \hline
          $A=1$ & 1 & 1 \\
          \hline
        \end{tabular}
      \end{center}

    \item The bottom left corresponds to $AB'$, the bottom right to $AB$, and the top right to $A'B$

    \end{itemize}

  \item Representation of $k$-Variable Functions

    \begin{itemize}

      \item Boolean Expression

      \item Truth Table

      \item Cube

      \item K Map

      \item Binary Decision Diagram

    \end{itemize}

  \item Boolean expressions can be minimized by combining terms

  \item K-Map Grouping Rules

    \begin{enumerate}

      \item Group all adjacent 1's

      \item Group only in powers of 2 (\textit{i.e.} $2^n$ cells)

      \item Use rectangles or squares to cover 1's in adjacent entries

      \item Rectangles can overlap but should not include 0's

      \item Groups should be as large as possible (avoid redundant groups)

      \item Adjacent cells extend to:

        \begin{itemize}

          \item Leftmost edge and rightmost edge are adjacent

          \item Top and bottom rows are adjacent

          \item The four corners are also adjacent

        \end{itemize}

    \end{enumerate}

    \begin{itemize}

      \item ``Don't Cares'' (\textsc{x}) don't matter at input or output, and can be assigned a 1 or 0 value

    \end{itemize}

\end{itemize}

\end{document}


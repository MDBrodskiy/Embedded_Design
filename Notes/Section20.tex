%%%%%%%%%%%%%%%%%%%%%%%%%%%%%%%%%%%%%%%%%%%%%%%%%%%%%%%%%%%%%%%%%%%%%%%%%%%%%%%%%%%%%%%%%%%%%%%%%%%%%%%%%%%%%%%%%%%%%%%%%%%%%%%%%%%%%%%%%%%%%%%%%%%%%%%%%%%%%%%%%%%
% Written By Michael Brodskiy
% Class: Embedded Design: Enabling Robotics
% Professor: S. Shazli
%%%%%%%%%%%%%%%%%%%%%%%%%%%%%%%%%%%%%%%%%%%%%%%%%%%%%%%%%%%%%%%%%%%%%%%%%%%%%%%%%%%%%%%%%%%%%%%%%%%%%%%%%%%%%%%%%%%%%%%%%%%%%%%%%%%%%%%%%%%%%%%%%%%%%%%%%%%%%%%%%%%

\include{Includes.tex}

\def\code#1{\texttt{#1}}

\pagestyle{fancy}

\title{Inheritance in C++}
\date{\today}
\author{Michael Brodskiy\\ \small Professor: S. Shazli}

\begin{document}

\maketitle

\thispagestyle{fancy}

\newpage

\begin{itemize}

  \item Inheritance

    \begin{itemize}

      \item The mechanism by which one class can acquire the properties of another class, and then extend that class

      \item We will utilize the ``is a'' relationship to define inheritance

        \begin{itemize}
            
          \item For example, a car is a vehicle

        \end{itemize}

    \end{itemize}

  \item Hierarchy

    \begin{itemize}

      \item Concepts at higher levels are more general

      \item Concepts at lower levels are more specific (inherit properties of concepts at higher levels)

      \item Derived classes are special cases of base classes

      \item A derived class can also serve as a base class for new classes

      \item There is no limit on the depth of inheritance allowed in C++ (as far as it is within the limits of the compiler)

      \item Derived classes can inherit from more than one base class

    \end{itemize}

\end{itemize}

\end{document}


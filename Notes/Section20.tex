%%%%%%%%%%%%%%%%%%%%%%%%%%%%%%%%%%%%%%%%%%%%%%%%%%%%%%%%%%%%%%%%%%%%%%%%%%%%%%%%%%%%%%%%%%%%%%%%%%%%%%%%%%%%%%%%%%%%%%%%%%%%%%%%%%%%%%%%%%%%%%%%%%%%%%%%%%%%%%%%%%%
% Written By Michael Brodskiy
% Class: Embedded Design: Enabling Robotics
% Professor: S. Shazli
%%%%%%%%%%%%%%%%%%%%%%%%%%%%%%%%%%%%%%%%%%%%%%%%%%%%%%%%%%%%%%%%%%%%%%%%%%%%%%%%%%%%%%%%%%%%%%%%%%%%%%%%%%%%%%%%%%%%%%%%%%%%%%%%%%%%%%%%%%%%%%%%%%%%%%%%%%%%%%%%%%%

\documentclass[12pt]{article} 
\usepackage{alphalph}
\usepackage[utf8]{inputenc}
\usepackage[russian,english]{babel}
\usepackage{titling}
\usepackage{amsmath}
\usepackage{graphicx}
\usepackage{enumitem}
\usepackage{amssymb}
\usepackage[super]{nth}
\usepackage{everysel}
\usepackage{ragged2e}
\usepackage{geometry}
\usepackage{multicol}
\usepackage{fancyhdr}
\usepackage{cancel}
\usepackage{siunitx}
\usepackage{physics}
\usepackage{lastpage}
\usepackage{tikz}
\usepackage{mathdots}
\usepackage{yhmath}
\usepackage{cancel}
\usepackage{color}
\usepackage{array}
\usepackage{multirow}
\usepackage{gensymb}
\usepackage{tabularx}
\usepackage{extarrows}
\usepackage{booktabs}
\usetikzlibrary{fadings}
\usetikzlibrary{patterns}
\usetikzlibrary{shadows.blur}
\usetikzlibrary{shapes}

\geometry{top=1.0in,bottom=1.0in,left=1.0in,right=1.0in}
\newcommand{\subtitle}[1]{%
  \posttitle{%
    \par\end{center}
    \begin{center}\large#1\end{center}
    \vskip0.5em}%

}
\usepackage{hyperref}
\hypersetup{
colorlinks=true,
linkcolor=blue,
filecolor=magenta,      
urlcolor=blue,
citecolor=blue,
}

\pagestyle{fancy}

\lfoot[\vspace{-15pt} \hline]{\vspace{-15pt} \hline}
\rfoot[\vspace{-15pt} \hline]{\vspace{-15pt} \hline}
\cfoot[\thepage]{\thepage}
\chead[\textsc{Embedded Systems}]{\textsc{Embedded Systems}}
\lhead[\textsc{EECE2160, CRN: 32014}]{\textsc{EECE2160, CRN: 32014}}
\rhead[\textsc{Page \thepage \hspace{1pt} of \pageref{LastPage}}]{\textsc{Page \thepage \hspace{1pt} of \pageref{LastPage}}}



\def\code#1{\texttt{#1}}

\pagestyle{fancy}

\title{Inheritance in C++}
\date{\today}
\author{Michael Brodskiy\\ \small Professor: S. Shazli}

\begin{document}

\maketitle

\thispagestyle{fancy}

\newpage

\begin{itemize}

  \item Inheritance

    \begin{itemize}

      \item The mechanism by which one class can acquire the properties of another class, and then extend that class

      \item We will utilize the ``is a'' relationship to define inheritance

        \begin{itemize}
            
          \item For example, a car is a vehicle

        \end{itemize}

    \end{itemize}

  \item Hierarchy

    \begin{itemize}

      \item Concepts at higher levels are more general

      \item Concepts at lower levels are more specific (inherit properties of concepts at higher levels)

      \item Derived classes are special cases of base classes

      \item A derived class can also serve as a base class for new classes

      \item There is no limit on the depth of inheritance allowed in C++ (as far as it is within the limits of the compiler)

      \item Derived classes can inherit from more than one base class

    \end{itemize}

  \item Three Benefits of Inheritance

    \begin{enumerate}

      \item You can reuse the methods and data of the existing class

      \item You can extend the existing class by adding new data and new methods

      \item You can modify the existing class by overloading its methods with your own implementations

    \end{enumerate}

  \item Protected and Private Inheritance

    \begin{itemize}

      \item With protected inheritance, public and protected members of $Y$ become protected in $X$ (\textit{i}.\textit{e}.\ classes derived from $X$ inherit the public members of $Y$ as protected)

      \item With private inheritance, public and protected members of $Y$ become private in $X$ (\textit{i}.\textit{e}.\ classes derived from $X$ inherit the public members of $Y$ as private)

      \item The default inheritance is private

    \end{itemize}

  \item Virtual Functions

    \begin{itemize}

      \item C++ uses virtual functions to implement run-time binding

      \item To force the compiler to generate code that guarantees dynamic binding, the word virtual should appear before the function declaration in the definition of the base class

    \end{itemize}

  \item \texttt{mmap()} and \texttt{munmap()}

    \begin{itemize}

      \item Used to allocate memory space by mapping to new address

      \item \texttt{mmap()} does the mapping

      \item \texttt{munmap()} does the opposite — clears a memory address

    \end{itemize}

  \item Certain devices can be mapped from physical to virtual space through memory mapping

\end{itemize}

\end{document}


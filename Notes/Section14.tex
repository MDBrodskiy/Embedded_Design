%%%%%%%%%%%%%%%%%%%%%%%%%%%%%%%%%%%%%%%%%%%%%%%%%%%%%%%%%%%%%%%%%%%%%%%%%%%%%%%%%%%%%%%%%%%%%%%%%%%%%%%%%%%%%%%%%%%%%%%%%%%%%%%%%%%%%%%%%%%%%%%%%%%%%%%%%%%%%%%%%%%
% Written By Michael Brodskiy
% Class: Embedded Design: Enabling Robotics
% Professor: S. Shazli
%%%%%%%%%%%%%%%%%%%%%%%%%%%%%%%%%%%%%%%%%%%%%%%%%%%%%%%%%%%%%%%%%%%%%%%%%%%%%%%%%%%%%%%%%%%%%%%%%%%%%%%%%%%%%%%%%%%%%%%%%%%%%%%%%%%%%%%%%%%%%%%%%%%%%%%%%%%%%%%%%%%

\include{Includes.tex}

\pagestyle{fancy}

\title{Finite State Machines}
\date{\today}
\author{Michael Brodskiy\\ \small Professor: S. Shazli}

\begin{document}

\maketitle

\thispagestyle{fancy}

\newpage

\begin{itemize}

\item State

  \begin{itemize}

    \item In order for a rotary lock to work, it has to keep track (remember) past events

    \item If the passcode is R13-L22-R3, the sequence of states to unlock is:

      \begin{enumerate}

        \item The lock is not open (locked), and no relevant operation has been performed

        \item Locked but user has completed R13

        \item Locked but user has completed R13-L22

        \item Unlocked: user has completed R13-L22-R3

      \end{itemize}

    \item A state diagram completely describes the operation of a sequential logic circuit

    \item Another example of state is a traffic light

    \item A standard Swiss traffic light has 4 states

      \begin{enumerate}

        \item Green

        \item Yellow

        \item Red

        \item Red and Yellow

      \end{enumerate}

    \item The sequence of these states is always as follows: $A\to B\to C\to D\to A$

    \item When should the light change from one state to another?

      \begin{itemize}

        \item We need a clock to indicate this

        \item At the start of a clock cycle, the system changes state

          \begin{itemize}

            \item During a clock cycle, the state always stays constant

          \end{itemize}

      \end{itemize}

  \end{itemize}

\item Changing State: The Notion of Clocks

  \begin{itemize}

    \item Clock is a general mechanism that triggers transition from one state to another in a sequential circuit

    \item Clock synchronizes state changes across many sequential circuit elements

    \item Combinational logic evaluates for the length of the clock cycle

    \item Clock cycle should be chosen to accommodate maximum combinational circuit delay

  \end{itemize}

\item Finite State Machines

  \begin{itemize}

    \item What is a Finite State Machine (FSM)?

      \begin{itemize}

        \item A discrete-time model of a stateful system

      \end{itemize}

    \item An FSM can model

      \begin{itemize}

        \item A traffic light, an elevator, fan speed, a microprocessor, etc.

      \end{itemize}

    \item Five Elements:

      \begin{itemize}

        \item A finite number of states

        \item A finite number of external inputs

        \item A finite number of external outputs

        \item An explicit specification of all state transitions

        \item An explicit specification of what determines state

      \end{itemize}

    \item Each FSM consists of three separate parts

      \begin{itemize}

        \item Next state logic

        \item State register

        \item Output logic

      \end{itemize}

    \item Moore vs. Mealy FSMs

      \begin{itemize}

        \item Next state is determined by the current state and inputs

        \item Two types of finite state machines differ in the output logic:

          \begin{itemize}

            \item Moore FSM: outputs depend only on the current state

            \item Mealy FSM: outputs depend only on the current state and the inputs

          \end{itemize}

      \end{itemize}

  \end{itemize}

\end{itemize}

\end{document}


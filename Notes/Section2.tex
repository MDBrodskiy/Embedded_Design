%%%%%%%%%%%%%%%%%%%%%%%%%%%%%%%%%%%%%%%%%%%%%%%%%%%%%%%%%%%%%%%%%%%%%%%%%%%%%%%%%%%%%%%%%%%%%%%%%%%%%%%%%%%%%%%%%%%%%%%%%%%%%%%%%%%%%%%%%%%%%%%%%%%%%%%%%%%%%%%%%%%
% Written By Michael Brodskiy
% Class: Embedded Design: Enabling Robotics
% Professor: S. Shazli
%%%%%%%%%%%%%%%%%%%%%%%%%%%%%%%%%%%%%%%%%%%%%%%%%%%%%%%%%%%%%%%%%%%%%%%%%%%%%%%%%%%%%%%%%%%%%%%%%%%%%%%%%%%%%%%%%%%%%%%%%%%%%%%%%%%%%%%%%%%%%%%%%%%%%%%%%%%%%%%%%%%

\include{Includes.tex}

\pagestyle{fancy}

\title{Number System Conversion\\Binary Arithmetic}
\date{\today}
\author{Michael Brodskiy\\ \small Professor: S. Shazli}

\begin{document}

\maketitle

\thispagestyle{fancy}

\begin{itemize}

  \item Binary Numbers

    \begin{itemize}

      \item All computers work with 0's and 1's so it is like learning alphabets before learning a language

    \end{itemize}

  \item Number Systems

    \begin{itemize}

      \item There is more than one way to express a number in binary, so 1010 could be -2, -5, or -6, and it is necessary to know which one

    \end{itemize}

  \item A/D and D/A Conversion

    \begin{itemize}

      \item Real world signals come in continuous/analog format, and it is good to how they become 0's and 1's (and vice versa)

    \end{itemize}

  \item Digital = Discrete

    \begin{itemize}

      \item Binary Codes

        \begin{itemize}

          \item Represent symbols using binary digits (bits)

        \end{itemize}

      \item Decimal digits 0-9

    \end{itemize}

  \item Digital Computers

    \begin{itemize}

      \item I/O is digital

      \item Internal representation is binary

      \item Sometimes use hexadecimal (base 16) for large binary numbers

    \end{itemize}

  \item Bases we will use

    \begin{itemize}

      \item Binary: Base 2

      \item Octal: Base 8

      \item Decimal: Base 10

      \item Hexadecimal: Base 16

    \end{itemize}

  \item Positional Number System

    \begin{itemize}

      \item $101_2=1\cdot2^2 + 0\cdot2^1 + 1\cdot2^0$

      \item $63_8=6\cdot 8^1 + 3\cdot8^0$

      \item $A1_{16}=10\cdot16^1 + 1\cdot16^0$

    \end{itemize}

  \item Conversion from binary to octal/hex

    \begin{itemize}

      \item Binary: 10011110001

        \item Octal: $10 | 011 | 110 | 001 = 2361_8$

        \item Hex: $100 | 1111 | 0001 = 4\text{F}1_{16}$

    \end{itemize}

  \item Conversion from binary to decimal

    \begin{itemize}

      \item $101_2=1\cdot2^2 + 0\cdot2^1 + 1\cdot2^0 = 5_{10}$

      \item $63.4_8=6\cdot8^1 + 3\cdot8^0+4\cdot8^{-1}=51.5_{10}$

      \item A$1_{16}=10\cdot16^1 + 1\cdot16^0=161_{10}$

    \end{itemize}

  \item Decimal to binary/octal/hex

    \begin{itemize}

      \item Binary: 56 becomes 111000$_2$

        \begin{center}
        \begin{tabular}[h]{c|c|c}
          & Quotient & Remainder\\
          \hline
          $56 / 2 =$ & 28 & 0\\
          $28 / 2 =$ & 14 & 0\\
          $14 / 2 =$ & 7 & 0\\
          $7 / 2 =$ & 3 & 1\\
          $3 / 2 =$ & 1 & 1\\
          $1 / 2 =$ & 0 & 1\\
        \end{tabular}
        \end{center}

    \item Octal: 56 becomes $70_8$

        \begin{center}
        \begin{tabular}[h]{c|c|c}
          & Quotient & Remainder\\
          \hline
          $56 / 8 =$ & 7 & 0\\
          $7 / 8 =$ & 0 & 7\\
        \end{tabular}
        \end{center}

      \item Each successive divide releases another LSB (least significant bit)

    \end{itemize}

  \item Negative Numbers

    \begin{itemize}

      \item Sign-and-magnitude

        \begin{itemize}

          \item The most-significant bit (MSB) is the sign digit, where 0 is positive and 1 is negative

          \item The remaining bits are the numbers magnitude

        \end{itemize}

      \item Ones-complement

        \begin{itemize}

          \item Negative number: a bitwise complement positive number

        \end{itemize}

      \item Twos-complement

        \begin{itemize}

          \item Most important option

          \item Simplifies arithmetic

          \item Used almost universally
            
          \item Negative number: a bitwise complement plus one

            \begin{itemize}

              \item $0011=3_{10}$

              \item $1101=-3_{10}$

            \end{itemize}

          \item Only one zero

          \item MSB is the sign digit, where 0 is positive and 1 is negative

          \item Overflow: summing two positive numbers gives a negative result and vice versa

        \end{itemize}

    \end{itemize}

  \item Signed-Complement Arithmetic

    \begin{itemize}

      \item Addition

        \begin{itemize}

          \item Add the numbers including the sign bits, discarding a carry out of the sign bits (2's Complement)

          \item If the sign bits were the same for both numbers and the sign of the result is different, an overflow has occurred

          \item The sign of the result is computed in step 1

        \end{itemize}

      \item Subtraction

        \begin{itemize}

          \item Form the complement of the number you are subtracting and follow the rules for addition
            
        \end{itemize}

    \end{itemize}

\end{itemize}

\end{document}


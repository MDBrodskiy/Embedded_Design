%%%%%%%%%%%%%%%%%%%%%%%%%%%%%%%%%%%%%%%%%%%%%%%%%%%%%%%%%%%%%%%%%%%%%%%%%%%%%%%%%%%%%%%%%%%%%%%%%%%%%%%%%%%%%%%%%%%%%%%%%%%%%%%%%%%%%%%%%%%%%%%%%%%%%%%%%%%%%%%%%%%
% Written By Michael Brodskiy
% Class: Embedded Design: Enabling Robotics
% Professor: S. Shazli
%%%%%%%%%%%%%%%%%%%%%%%%%%%%%%%%%%%%%%%%%%%%%%%%%%%%%%%%%%%%%%%%%%%%%%%%%%%%%%%%%%%%%%%%%%%%%%%%%%%%%%%%%%%%%%%%%%%%%%%%%%%%%%%%%%%%%%%%%%%%%%%%%%%%%%%%%%%%%%%%%%%

\documentclass[12pt]{article} 
\usepackage{alphalph}
\usepackage[utf8]{inputenc}
\usepackage[russian,english]{babel}
\usepackage{titling}
\usepackage{amsmath}
\usepackage{graphicx}
\usepackage{enumitem}
\usepackage{amssymb}
\usepackage[super]{nth}
\usepackage{everysel}
\usepackage{ragged2e}
\usepackage{geometry}
\usepackage{multicol}
\usepackage{fancyhdr}
\usepackage{cancel}
\usepackage{siunitx}
\usepackage{physics}
\usepackage{lastpage}
\usepackage{tikz}
\usepackage{mathdots}
\usepackage{yhmath}
\usepackage{cancel}
\usepackage{color}
\usepackage{array}
\usepackage{multirow}
\usepackage{gensymb}
\usepackage{tabularx}
\usepackage{extarrows}
\usepackage{booktabs}
\usetikzlibrary{fadings}
\usetikzlibrary{patterns}
\usetikzlibrary{shadows.blur}
\usetikzlibrary{shapes}

\geometry{top=1.0in,bottom=1.0in,left=1.0in,right=1.0in}
\newcommand{\subtitle}[1]{%
  \posttitle{%
    \par\end{center}
    \begin{center}\large#1\end{center}
    \vskip0.5em}%

}
\usepackage{hyperref}
\hypersetup{
colorlinks=true,
linkcolor=blue,
filecolor=magenta,      
urlcolor=blue,
citecolor=blue,
}

\pagestyle{fancy}

\lfoot[\vspace{-15pt} \hline]{\vspace{-15pt} \hline}
\rfoot[\vspace{-15pt} \hline]{\vspace{-15pt} \hline}
\cfoot[\thepage]{\thepage}
\chead[\textsc{Embedded Systems}]{\textsc{Embedded Systems}}
\lhead[\textsc{EECE2160, CRN: 32014}]{\textsc{EECE2160, CRN: 32014}}
\rhead[\textsc{Page \thepage \hspace{1pt} of \pageref{LastPage}}]{\textsc{Page \thepage \hspace{1pt} of \pageref{LastPage}}}



\def\code#1{\texttt{#1}}

\pagestyle{fancy}

\title{Classes in C++}
\date{\today}
\author{Michael Brodskiy\\ \small Professor: S. Shazli}

\begin{document}

\maketitle

\thispagestyle{fancy}

\newpage

\begin{itemize}

  \item The \texttt{const} modifier forbids the compiler from modifying any variables

    \begin{itemize}

      \item This is common practice to prevent modification of variables in large code bases

    \end{itemize}

  \item Getters are also known as accessors

  \item The \texttt{virtual} keyword

  \item A \texttt{static} variable is the same across different instances of a class

  \item Suppose that:

    \begin{itemize}

      \item You want to write a function to compare two ints

      \item You want to write a function to compare two strings

        \begin{itemize}

          \item Function overloading

        \end{itemize}

    \end{itemize}

  \item Polymorphism in C++

    \begin{itemize}

      \item In C++: \texttt{PromisedType* var\_p = new ActualType();}

      \item C++ has the notion of templates

        \begin{itemize}

          \item A function or class that accepts a \texttt{type} as a parameter

          \item Supposing \texttt{template <typename T>} is passed to a function, the function may be used by describing the data type passed while passing it:

            \begin{itemize}

              \item For an int: \texttt{function<int>(nameOfInt)}

              \item For a double: \texttt{function<double>(nameOfDouble)}

              \item For a string: \texttt{function<std::string>(nameOfString)}

            \end{itemize}

        \end{itemize}

      \item The compiler doesn't generate any code when it sees the template function

        \begin{itemize}

          \item It doesn't know what code to generate yet, since it doesn't know what types are involved

        \end{itemize}

      \item When the compiler sees the function being used, it then understands what types are involved

    \end{itemize}

  \item Templates are useful for classes as well

    \begin{itemize}

      \item One of the main motivations of templates

      \item Imagine we want a class that holds a pair of things that we can set and get the value of, but we don't know what data type the things will be

      \item Thing is replaced with template argument when the class is instantiated

        \begin{itemize}

          \item The class template parameter name is in scope of the template class definition and can be freely used there

          \item Class template member functions are template functions with template parameters that match those of the class template

            \begin{itemize}

              \item These member functions must be defined as template functions outside of the class template definition (if not written inline)

              \item The template parameter name does not need to match that used in the template class definition

            \end{itemize}

        \end{itemize}

    \end{itemize}

  \item Abstract Methods and Classes

    \begin{itemize}

      \item Sometimes we want to include a function in a class but only implement it in derived classes

        \begin{itemize}

          \item In Java, we would use an abstract method

          \item In C++, we use a ``pure virtual'' function

          \item For example, \texttt{string noise() = 0}

          \item A class containing any pure virtual methods is abstract

            \begin{itemize}

              \item You can't create instances of an abstract class

              \item Extend abstract classes and override methods to use them

            \end{itemize}
            
          \item A class containing only pure virtual methods is the same as a Java interface

            \begin{itemize}

              \item Pure type specification without implementations

            \end{itemize}

        \end{itemize}

    \end{itemize}

\end{itemize}

\end{document}


%%%%%%%%%%%%%%%%%%%%%%%%%%%%%%%%%%%%%%%%%%%%%%%%%%%%%%%%%%%%%%%%%%%%%%%%%%%%%%%%%%%%%%%%%%%%%%%%%%%%%%%%%%%%%%%%%%%%%%%%%%%%%%%%%%%%%%%%%%%%%%%%%%%%%%%%%%%%%%%%%%%
% Written By Michael Brodskiy
% Class: Embedded Design: Enabling Robotics
% Professor: S. Shazli
%%%%%%%%%%%%%%%%%%%%%%%%%%%%%%%%%%%%%%%%%%%%%%%%%%%%%%%%%%%%%%%%%%%%%%%%%%%%%%%%%%%%%%%%%%%%%%%%%%%%%%%%%%%%%%%%%%%%%%%%%%%%%%%%%%%%%%%%%%%%%%%%%%%%%%%%%%%%%%%%%%%

\documentclass[12pt]{article} 
\usepackage{alphalph}
\usepackage[utf8]{inputenc}
\usepackage[russian,english]{babel}
\usepackage{titling}
\usepackage{amsmath}
\usepackage{graphicx}
\usepackage{enumitem}
\usepackage{amssymb}
\usepackage[super]{nth}
\usepackage{everysel}
\usepackage{ragged2e}
\usepackage{geometry}
\usepackage{multicol}
\usepackage{fancyhdr}
\usepackage{cancel}
\usepackage{siunitx}
\usepackage{physics}
\usepackage{lastpage}
\usepackage{tikz}
\usepackage{mathdots}
\usepackage{yhmath}
\usepackage{cancel}
\usepackage{color}
\usepackage{array}
\usepackage{multirow}
\usepackage{gensymb}
\usepackage{tabularx}
\usepackage{extarrows}
\usepackage{booktabs}
\usetikzlibrary{fadings}
\usetikzlibrary{patterns}
\usetikzlibrary{shadows.blur}
\usetikzlibrary{shapes}

\geometry{top=1.0in,bottom=1.0in,left=1.0in,right=1.0in}
\newcommand{\subtitle}[1]{%
  \posttitle{%
    \par\end{center}
    \begin{center}\large#1\end{center}
    \vskip0.5em}%

}
\usepackage{hyperref}
\hypersetup{
colorlinks=true,
linkcolor=blue,
filecolor=magenta,      
urlcolor=blue,
citecolor=blue,
}

\pagestyle{fancy}

\lfoot[\vspace{-15pt} \hline]{\vspace{-15pt} \hline}
\rfoot[\vspace{-15pt} \hline]{\vspace{-15pt} \hline}
\cfoot[\thepage]{\thepage}
\chead[\textsc{Embedded Systems}]{\textsc{Embedded Systems}}
\lhead[\textsc{EECE2160, CRN: 32014}]{\textsc{EECE2160, CRN: 32014}}
\rhead[\textsc{Page \thepage \hspace{1pt} of \pageref{LastPage}}]{\textsc{Page \thepage \hspace{1pt} of \pageref{LastPage}}}



\def\code#1{\texttt{#1}}

\pagestyle{fancy}

\title{Makefiles}
\date{\today}
\author{Michael Brodskiy\\ \small Professor: S. Shazli}

\begin{document}

\maketitle

\thispagestyle{fancy}

\newpage

\begin{itemize}

  \item Header files consist of definitions of functions and declarations of variables

  \item \texttt{\#ifndef} is used to define something in the code if it is not already defined (usually used for constants)

  \item \texttt{\#define} is used to just define something

  \item When compiling multiple files, the following rules apply:

    \begin{itemize}

      \item Compiled:

        \begin{itemize}

          \item .c or .cpp

        \end{itemize}

      \item Not Compiled:

        \begin{itemize}

          \item .h or .hpp

        \end{itemize}

    \end{itemize}

  \item \texttt{make} is a tool that is designed to allow programmers to efficiently compile large complex programs with many components easily

  \item The \texttt{make} utility allows us to only compile those that have changed and the modules that depend upon them

  \item If any of the associated files have been modified, then it recompiles

  \item If the file \texttt{<target>} does not exist, or the dependency files are younger then execute \texttt{<buildCommand>}

  \item To run a makefile and compile a program, we run the command \texttt{make}

  \item The \texttt{make} command will build the first target which is the only target in this example

  \item If you want to build a specific target if there are multiple targets, you can specify the target with the \texttt{make} command

  \item A \texttt{clean} target in a makefile removes or cleans the system of specified files

    \begin{itemize}

      \item Executed using \texttt{make clean}

    \end{itemize}

  \item Variables/symbols may be used to make the makefile easier to construct

    \begin{itemize}

      \item For example, use something like \texttt{PROJECT = homework2} as a variable, and then \texttt{\$(PROJECT)} to refer to it

      \item ``Phony'' targets do not create files, will always be executed when called

      \item Flags and compiler may be specified as well

    \end{itemize}

  \item A makefile will be necessary for the final project

  \item The \texttt{LEDControl} class from the previous lab will be modified to consist of multiple files, which create an object \texttt{DE1Socfpga} with a header file

  \item This will then be applied to seven segment displays

  \item The displays will be controlled by objects with header files as well

  \item Bit manipulation will be necessary to control the displays

\end{itemize}

\end{document}


%%%%%%%%%%%%%%%%%%%%%%%%%%%%%%%%%%%%%%%%%%%%%%%%%%%%%%%%%%%%%%%%%%%%%%%%%%%%%%%%%%%%%%%%%%%%%%%%%%%%%%%%%%%%%%%%%%%%%%%%%%%%%%%%%%%%%%%%%%%%%%%%%%%%%%%%%%%%%%%%%%%
% Written By Michael Brodskiy
% Class: Embedded Design: Enabling Robotics
% Professor: S. Shazli
%%%%%%%%%%%%%%%%%%%%%%%%%%%%%%%%%%%%%%%%%%%%%%%%%%%%%%%%%%%%%%%%%%%%%%%%%%%%%%%%%%%%%%%%%%%%%%%%%%%%%%%%%%%%%%%%%%%%%%%%%%%%%%%%%%%%%%%%%%%%%%%%%%%%%%%%%%%%%%%%%%%

\documentclass[12pt]{article} 
\usepackage{alphalph}
\usepackage[utf8]{inputenc}
\usepackage[russian,english]{babel}
\usepackage{titling}
\usepackage{amsmath}
\usepackage{graphicx}
\usepackage{enumitem}
\usepackage{amssymb}
\usepackage[super]{nth}
\usepackage{everysel}
\usepackage{ragged2e}
\usepackage{geometry}
\usepackage{multicol}
\usepackage{fancyhdr}
\usepackage{cancel}
\usepackage{siunitx}
\usepackage{physics}
\usepackage{lastpage}
\usepackage{tikz}
\usepackage{mathdots}
\usepackage{yhmath}
\usepackage{cancel}
\usepackage{color}
\usepackage{array}
\usepackage{multirow}
\usepackage{gensymb}
\usepackage{tabularx}
\usepackage{extarrows}
\usepackage{booktabs}
\usetikzlibrary{fadings}
\usetikzlibrary{patterns}
\usetikzlibrary{shadows.blur}
\usetikzlibrary{shapes}

\geometry{top=1.0in,bottom=1.0in,left=1.0in,right=1.0in}
\newcommand{\subtitle}[1]{%
  \posttitle{%
    \par\end{center}
    \begin{center}\large#1\end{center}
    \vskip0.5em}%

}
\usepackage{hyperref}
\hypersetup{
colorlinks=true,
linkcolor=blue,
filecolor=magenta,      
urlcolor=blue,
citecolor=blue,
}

\pagestyle{fancy}

\lfoot[\vspace{-15pt} \hline]{\vspace{-15pt} \hline}
\rfoot[\vspace{-15pt} \hline]{\vspace{-15pt} \hline}
\cfoot[\thepage]{\thepage}
\chead[\textsc{Embedded Systems}]{\textsc{Embedded Systems}}
\lhead[\textsc{EECE2160, CRN: 32014}]{\textsc{EECE2160, CRN: 32014}}
\rhead[\textsc{Page \thepage \hspace{1pt} of \pageref{LastPage}}]{\textsc{Page \thepage \hspace{1pt} of \pageref{LastPage}}}



\pagestyle{fancy}

\title{Digital Logic Circuits Cont'd}
\date{\today}
\author{Michael Brodskiy\\ \small Professor: S. Shazli}

\begin{document}

\maketitle

\thispagestyle{fancy}

\newpage

\begin{itemize}

  \item Logical Equivalences of And

    \begin{itemize}

      \item $(p\wedge q) \wedge r \equiv p\wedge (q\wedge r)$ — Associative Law

        \begin{center}
          \begin{tabular}[h!]{|c|c|c|c|c|c|c|}
            \hline
            $p$ & $q$ & $r$ & $p\wedge q$ & $(p\wedge q)\wedge r$ & $q\wedge r$ & $p\wedge (q\wedge r)$\\
            \hline
            T & T & T & T & T & T & T\\
            \hline
            T & T & F & T & F & F & F\\
            \hline
            T & F & T & F & F & F & F\\
            \hline
            T & F & F & F & F & F & F\\
            \hline
            F & T & T & F & F & T & F\\
            \hline
            F & T & F & F & F & F & F\\
            \hline
            F & F & T & F & F & F & F\\
            \hline
            F & F & F & F & F & F & F\\
            \hline
          \end{tabular}
        \end{center}

    \end{itemize}

  \item Logical Equivalences of Or

    \begin{itemize}

      \item $p\vee T\equiv T$ — Identity Law

      \item $p\vee F\equiv p$ — Domination Law

      \item $p\vee p\equiv p$ — Idempotent Law

      \item $p\vee q\equiv q\vee p$ — Commutative Law

      \item $(p \vee q )\vee r \equiv p\vee (q\vee r)$ — Associative Law

      \item $\sim (p\vee q) \equiv (\sim p)\wedge (\sim q)$ — De Morgan's Law

    \end{itemize}

  \item Can implement any truth table with AND, OR, and NOT

    \begin{enumerate}

      \item AND combinations that yield a ``1'' in the truth table

      \item OR the results of the AND gates

    \end{enumerate}

  \item Sum of Products Form: Key Idea

    \begin{itemize}

      \item Assume we have the truth table of a boolean function

      \item How do we express the function in terms of the inputs in a standard manner?

        \begin{itemize}

          \item Idea: Sum of Products Form

          \item Express the truth table as a two-level Boolean expression

          \item If ANY of the combinations of input variables that results in a 1 is TRUE, then the output is 1

          \item F = OR of all input variable combinations that result in a 1

        \end{itemize}

      \item Complement: Variable with a bar over it

        $$\bar{A},\,\bar{B},\,\bar{C}$$

      \item Literal: Variable or its complement

        $$A,\,\bar{A},\,B,\,\bar{B},\,C,\,\bar{C}$$

      \item Implicant: Product (AND) of literals

        $$(A\bullet B\bullet\bar{C}),\,(\bar{A}\bullet C),\,(B\bullet\bar{C})$$

      \item Minterm: Product (AND) that includes all input variables

        $$(A\bullet B\bullet \bar{C}),\,(\bar{A}\bullet\bar{B}\bullet C),\,(\bar{A}\bullet B\bullet\bar{C})$$

      \item Maxterm: Sum (OR) that includes all input variables

        $$(A\vee B\vee \bar{C}),\,(\bar{A}\vee\bar{B}\vee C),\,(\bar{A}\vee B\vee\bar{C})$$

    \end{itemize}

  \item Canonical Form: Standard form for a boolean expression

  \item Sum of Products Form (SOP)

    \begin{itemize}

      \item Also known as a disjunctive normal form or minterm expansion

      \item Each row in a truth table has a minterm

      \item A minterm is a product (AND) of literals

      \item Each minterm is TRUE for that row (and only that row)

      \item All boolean equations can be written in SOP form

      \item Standard ``shorthand'' notation

        \begin{itemize}

          \item If the order of variables in the rows of a truth table are agreed upon, then one may write \textsc{m<row>} as shorthand, for example, \textsc{m4} for row 4

          \item This can be be rewritten as a sum of products or summation notation:

            $$m3 + m4 + m5 + m6 + m7=\sum m(3,4,5,6,7)$$

          \item The canonical form is not always the minimal form

        \end{itemize}

    \end{itemize}

  \item Product of Sums (POS) Form

    \begin{itemize}

      \item Each sum term represents one of the ``zeros'' of the function

      \item Write the inverses of the zero functions as each term

        \begin{itemize}

          \item Ex. If a truth table gives $\begin{array}{c c c} A & B & C\\ 0 & 1 & 0 \end{array}$, then a term of the product of sums would $(A\vee \bar{B}\vee C)$

          \item Multiply (AND) each of these terms

        \end{itemize}

      \item This is also known as maxterm form or conjunctive normal form

        \begin{enumerate}

          \item Find truth table rows where $F$ is 0

          \item $0$ in input column $\rightarrow$ true literal

          \item $1$ in input column $\rightarrow$ complemented literal

          \item OR the literals to get a maxterm

          \item AND together all the maxterms

        \end{enumerate}

      \item Standard ``shorthand'' notation

        \begin{itemize}

          \item If the order of variables in the rows of a truth table are agreed upon, then one may write \textsc{M<row>} as shorthand, for example, \textsc{M4} for row 4

          \item This can be be rewritten as a product of the sums or product notation:

            $$(M0)(M1)(M2)=\prod M(0,1,2)$$

          \item The canonical form is not always the minimal form

        \end{itemize}

    \end{itemize}

  \item Minterm-Maxterm Conversion

    \begin{itemize}

      \item Rewrite minterm shorthand using maxterm shorthand, replacing minterm indices with the indices not already used, and vice versa:

        $$\sum m(3,4,5,6,7)=\prod M(0,1,2)$$

    \end{itemize}

\end{itemize}

\end{document}


%%%%%%%%%%%%%%%%%%%%%%%%%%%%%%%%%%%%%%%%%%%%%%%%%%%%%%%%%%%%%%%%%%%%%%%%%%%%%%%%%%%%%%%%%%%%%%%%%%%%%%%%%%%%%%%%%%%%%%%%%%%%%%%%%%%%%%%%%%%%%%%%%%%%%%%%%%%%%%%%%%%
% Written By Michael Brodskiy
% Class: Embedded Design: Enabling Robotics
% Professor: S. Shazli
%%%%%%%%%%%%%%%%%%%%%%%%%%%%%%%%%%%%%%%%%%%%%%%%%%%%%%%%%%%%%%%%%%%%%%%%%%%%%%%%%%%%%%%%%%%%%%%%%%%%%%%%%%%%%%%%%%%%%%%%%%%%%%%%%%%%%%%%%%%%%%%%%%%%%%%%%%%%%%%%%%%

\documentclass[12pt]{article} 
\usepackage{alphalph}
\usepackage[utf8]{inputenc}
\usepackage[russian,english]{babel}
\usepackage{titling}
\usepackage{amsmath}
\usepackage{graphicx}
\usepackage{enumitem}
\usepackage{amssymb}
\usepackage[super]{nth}
\usepackage{everysel}
\usepackage{ragged2e}
\usepackage{geometry}
\usepackage{multicol}
\usepackage{fancyhdr}
\usepackage{cancel}
\usepackage{siunitx}
\usepackage{physics}
\usepackage{lastpage}
\usepackage{tikz}
\usepackage{mathdots}
\usepackage{yhmath}
\usepackage{cancel}
\usepackage{color}
\usepackage{array}
\usepackage{multirow}
\usepackage{gensymb}
\usepackage{tabularx}
\usepackage{extarrows}
\usepackage{booktabs}
\usetikzlibrary{fadings}
\usetikzlibrary{patterns}
\usetikzlibrary{shadows.blur}
\usetikzlibrary{shapes}

\geometry{top=1.0in,bottom=1.0in,left=1.0in,right=1.0in}
\newcommand{\subtitle}[1]{%
  \posttitle{%
    \par\end{center}
    \begin{center}\large#1\end{center}
    \vskip0.5em}%

}
\usepackage{hyperref}
\hypersetup{
colorlinks=true,
linkcolor=blue,
filecolor=magenta,      
urlcolor=blue,
citecolor=blue,
}

\pagestyle{fancy}

\lfoot[\vspace{-15pt} \hline]{\vspace{-15pt} \hline}
\rfoot[\vspace{-15pt} \hline]{\vspace{-15pt} \hline}
\cfoot[\thepage]{\thepage}
\chead[\textsc{Embedded Systems}]{\textsc{Embedded Systems}}
\lhead[\textsc{EECE2160, CRN: 32014}]{\textsc{EECE2160, CRN: 32014}}
\rhead[\textsc{Page \thepage \hspace{1pt} of \pageref{LastPage}}]{\textsc{Page \thepage \hspace{1pt} of \pageref{LastPage}}}



\def\code#1{\texttt{#1}}

\pagestyle{fancy}

\title{C++ Functions}
\date{\today}
\author{Michael Brodskiy\\ \small Professor: S. Shazli}

\begin{document}

\maketitle

\thispagestyle{fancy}

\newpage

\begin{itemize}

  \item Writing a \code{swap} function

    \begin{itemize}

      \item If we want to swap the values \code{a} and \code{b}, we need an intermediate value to hold the value as a transition

    \end{itemize}

  \item C++ offers an alternative parameter-passing method called pass-by-reference (using the \& operator)

  \item When we pass by reference, the data being passed is the address of the argument, not the argument itself

  \item Memory Layout

    \begin{itemize}

      \item Text: Program code

      \item Data: Global variables

      \item BSS: Global and static variables

      \item Stack: Local variables

      \item Heap: Dynamic memory

    \end{itemize}

  \item When is memory allocated?

    \begin{itemize}

      \item Global and static: Upon program start

      \item Local variables: Upon function call

      \item Dynamic memory: \code{new} keyword call (C++)

    \end{itemize}

  \item Stack

    \begin{itemize}

      \item Memory for C/C++ run-time system to keep track of active functions

        \begin{itemize}

          \item Stack pointer (SP)

        \end{itemize}

    \end{itemize}

  \item Dynamic Memory Allocation

    \begin{itemize}

      \item Operator \code{new} is used to request memory space enough to hold a specific data type or an array of the data type

    \end{itemize}

  \item \code{struct} is much like an array

    \begin{itemize}

      \item The structure stores multiple data

        \begin{itemize}

          \item You can access the individual data, or you can reference the entire structure

        \end{itemize}

      \item To access a particular member, you use the \code{.} operator

        \begin{itemize}

          \item As in \code{student.firstName} or \code{p1.x}

            \begin{itemize}

              \item We will see later that we will also use \code{->} to reference a field if the \code{struct} is pointed to by a pointer

            \end{itemize}

        \end{itemize}

      \item We may pass \code{struct}s as parameters

      \item The parameter would be entered as \code{struct} \code{<tag> <name>}

      \item Passing a struct by value has two flaws:

        \begin{itemize}

          \item Twice as much memory is required

          \item It requires copying each member of temp back into the members of the original \code{struct}

          \item A pointer may be used instead

        \end{itemize}

    \end{itemize}

\end{itemize}

\end{document}


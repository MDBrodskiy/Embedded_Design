%%%%%%%%%%%%%%%%%%%%%%%%%%%%%%%%%%%%%%%%%%%%%%%%%%%%%%%%%%%%%%%%%%%%%%%%%%%%%%%%%%%%%%%%%%%%%%%%%%%%%%%%%%%%%%%%%%%%%%%%%%%%%%%%%%%%%%%%%%%%%%%%%%%%%%%%%%%%%%%%%%%
% Written By Michael Brodskiy
% Class: Embedded Design: Enabling Robotics
% Professor: S. Shazli
%%%%%%%%%%%%%%%%%%%%%%%%%%%%%%%%%%%%%%%%%%%%%%%%%%%%%%%%%%%%%%%%%%%%%%%%%%%%%%%%%%%%%%%%%%%%%%%%%%%%%%%%%%%%%%%%%%%%%%%%%%%%%%%%%%%%%%%%%%%%%%%%%%%%%%%%%%%%%%%%%%%

\documentclass[12pt]{article} 
\usepackage{alphalph}
\usepackage[utf8]{inputenc}
\usepackage[russian,english]{babel}
\usepackage{titling}
\usepackage{amsmath}
\usepackage{graphicx}
\usepackage{enumitem}
\usepackage{amssymb}
\usepackage[super]{nth}
\usepackage{everysel}
\usepackage{ragged2e}
\usepackage{geometry}
\usepackage{multicol}
\usepackage{fancyhdr}
\usepackage{cancel}
\usepackage{siunitx}
\usepackage{physics}
\usepackage{lastpage}
\usepackage{tikz}
\usepackage{mathdots}
\usepackage{yhmath}
\usepackage{cancel}
\usepackage{color}
\usepackage{array}
\usepackage{multirow}
\usepackage{gensymb}
\usepackage{tabularx}
\usepackage{extarrows}
\usepackage{booktabs}
\usetikzlibrary{fadings}
\usetikzlibrary{patterns}
\usetikzlibrary{shadows.blur}
\usetikzlibrary{shapes}

\geometry{top=1.0in,bottom=1.0in,left=1.0in,right=1.0in}
\newcommand{\subtitle}[1]{%
  \posttitle{%
    \par\end{center}
    \begin{center}\large#1\end{center}
    \vskip0.5em}%

}
\usepackage{hyperref}
\hypersetup{
colorlinks=true,
linkcolor=blue,
filecolor=magenta,      
urlcolor=blue,
citecolor=blue,
}

\pagestyle{fancy}

\lfoot[\vspace{-15pt} \hline]{\vspace{-15pt} \hline}
\rfoot[\vspace{-15pt} \hline]{\vspace{-15pt} \hline}
\cfoot[\thepage]{\thepage}
\chead[\textsc{Embedded Systems}]{\textsc{Embedded Systems}}
\lhead[\textsc{EECE2160, CRN: 32014}]{\textsc{EECE2160, CRN: 32014}}
\rhead[\textsc{Page \thepage \hspace{1pt} of \pageref{LastPage}}]{\textsc{Page \thepage \hspace{1pt} of \pageref{LastPage}}}



\pagestyle{fancy}

\title{Multiplexers, Demultiplexers, Decoders, and Encoders}
\date{\today}
\author{Michael Brodskiy\\ \small Professor: S. Shazli}

\begin{document}

\maketitle

\thispagestyle{fancy}

\newpage

\begin{itemize}

  \item A demultiplexer has:

    \begin{itemize}

      \item $N$ control inputs

      \item 1 data input

      \item $2^N$ outputs

    \end{itemize}

  \item A demultiplexer routes (or connects) the data input to the selected output

    \begin{itemize}

      \item The value of the control inputs determines the output that is selected

    \end{itemize}

  \item A demultiplexer performs the opposite function of a multiplexer

  \item Using an $n$-input multiplexer

    \begin{itemize}

      \item Use an $n$-input multiplexer to realize a logic circuit for a function with $n$ minterms

        \begin{itemize}

          \item $n=2^m$, where $m=$ \# of variables in the function

        \end{itemize}

      \item Each minterm of the function can be mapped to an input of the multiplexer

      \item For each row in the truth table, for the function, where the output is 1, set the corresponding input of the multiplexer to 1

        \begin{itemize}

          \item That is, for each minterm in the minterm expansion of the function, set the corresponding input of the multiplexer to 1

        \end{itemize}

      \item Set the remaining inputs of the multiplexer to 0

    \end{itemize}

  \item Using an ($\frac{n}{2}$)-input multiplexer

    \begin{itemize}

      \item $n=2^m$, wehere $m =$ the number of variables in the function

    \end{itemize}

  \item Group the rows of the truth table, for the function, into $\dfrac{n}{2}$ pairs of rows

    \begin{itemize}

      \item Each pair of rows represents a product term of ($m-1$) variables

      \item Each pair of rows can be mapped to a multiplexer input

    \end{itemize}

  \item Determine the logical function of each pair of rows in terms of the $m^{th}$ variable

    \begin{itemize}

      \item If the $m^{th}$ variable, for example, is $x$, then the possible values are $x$, $x'$, 0, and 1

    \end{itemize}

  \item Decoders

    \begin{itemize}

      \item A decoder has

        \begin{itemize}

          \item $N$ inputs

          \item $2^N$ outputs

        \end{itemize}

      \item A decoder selects one of $2^N$ outputs by decoding the binary value on the $N$ inputs

      \item The decoder generates all of the minterms of the N input variables

        \begin{itemize}

          \item Exactly one output will be active for each combination of the inputs

        \end{itemize}

    \end{itemize}

  \item Encoders

    \begin{itemize}

      \item An encoder has:

        \begin{itemize}

          \item $2^N$ inputs

          \item $N$ outputs

        \end{itemize}

      \item An encoder outputs the binary value of the selected (or active) input

      \item An encoder performs the inverse operation of a decoder

    \end{itemize}

\end{itemize}

\end{document}


%%%%%%%%%%%%%%%%%%%%%%%%%%%%%%%%%%%%%%%%%%%%%%%%%%%%%%%%%%%%%%%%%%%%%%%%%%%%%%%%%%%%%%%%%%%%%%%%%%%%%%%%%%%%%%%%%%%%%%%%%%%%%%%%%%%%%%%%%%%%%%%%%%%%%%%%%%%%%%%%%%%
% Written By Michael Brodskiy
% Class: Embedded Design: Enabling Robotics
% Professor: S. Shazli
%%%%%%%%%%%%%%%%%%%%%%%%%%%%%%%%%%%%%%%%%%%%%%%%%%%%%%%%%%%%%%%%%%%%%%%%%%%%%%%%%%%%%%%%%%%%%%%%%%%%%%%%%%%%%%%%%%%%%%%%%%%%%%%%%%%%%%%%%%%%%%%%%%%%%%%%%%%%%%%%%%%

\documentclass[12pt]{article} 
\usepackage{alphalph}
\usepackage[utf8]{inputenc}
\usepackage[russian,english]{babel}
\usepackage{titling}
\usepackage{amsmath}
\usepackage{graphicx}
\usepackage{enumitem}
\usepackage{amssymb}
\usepackage[super]{nth}
\usepackage{everysel}
\usepackage{ragged2e}
\usepackage{geometry}
\usepackage{multicol}
\usepackage{fancyhdr}
\usepackage{cancel}
\usepackage{siunitx}
\usepackage{physics}
\usepackage{lastpage}
\usepackage{tikz}
\usepackage{mathdots}
\usepackage{yhmath}
\usepackage{cancel}
\usepackage{color}
\usepackage{array}
\usepackage{multirow}
\usepackage{gensymb}
\usepackage{tabularx}
\usepackage{extarrows}
\usepackage{booktabs}
\usetikzlibrary{fadings}
\usetikzlibrary{patterns}
\usetikzlibrary{shadows.blur}
\usetikzlibrary{shapes}

\geometry{top=1.0in,bottom=1.0in,left=1.0in,right=1.0in}
\newcommand{\subtitle}[1]{%
  \posttitle{%
    \par\end{center}
    \begin{center}\large#1\end{center}
    \vskip0.5em}%

}
\usepackage{hyperref}
\hypersetup{
colorlinks=true,
linkcolor=blue,
filecolor=magenta,      
urlcolor=blue,
citecolor=blue,
}

\pagestyle{fancy}

\lfoot[\vspace{-15pt} \hline]{\vspace{-15pt} \hline}
\rfoot[\vspace{-15pt} \hline]{\vspace{-15pt} \hline}
\cfoot[\thepage]{\thepage}
\chead[\textsc{Embedded Systems}]{\textsc{Embedded Systems}}
\lhead[\textsc{EECE2160, CRN: 32014}]{\textsc{EECE2160, CRN: 32014}}
\rhead[\textsc{Page \thepage \hspace{1pt} of \pageref{LastPage}}]{\textsc{Page \thepage \hspace{1pt} of \pageref{LastPage}}}



\def\code#1{\texttt{#1}}

\pagestyle{fancy}

\title{C++}
\date{\today}
\author{Michael Brodskiy\\ \small Professor: S. Shazli}

\begin{document}

\maketitle

\thispagestyle{fancy}

\newpage

\begin{itemize}

  \item Headers are included at the top, and are denoted by \code{\#include}

    \begin{itemize}

      \item We will almost always be defining a header for \code{cin} and \code{cout} (for input and output)

    \end{itemize}

  \item After you write a C++ program, you compile it; that is, you run a program called a compiler that checks whether the program follows the C++ syntax

    \begin{itemize}

      \item If it finds errors, it lists them

      \item If there are no errors, it translates the C++ program into machine language which can be executed

    \end{itemize}

  \item Single-line comments begin with //

  \item Indentation is for the convenience of the reader

    \begin{itemize}

      \item The compiler ignores white space

    \end{itemize}

  \item Input statements would begin with \code{cin >> a}, where \code{a} would be some kind of input, like a variable

  \item Output statements begin with \code{cout << a}, where \code{a} would be some kind of output, like a String of text

  \item Functions

    \begin{itemize}

      \item C++ functions are specialized blocks

      \item Each one begins with a return type, function name, and input parameters, in the following format:

        \begin{center}
          \code{<return type> <name>(<params>) \{ \}}
        \end{center}

      \item All functions should be declared before main

      \item Function names are generally camel-case (starts with lowercase, and every subsequent word is capitalized)

    \end{itemize}

  \item Always put comments in the code

    \begin{itemize}

      \item Start with a multi-line comment with author information
        
      \item Multi-line comments are denoted with \code{/*} and \code{*/}

    \end{itemize}

  \item Arrays and Pointers

    \begin{itemize}

      \item A pointer is merely an address of where a datum or structure is stored

        \begin{itemize}

          \item All pointers are typed based on the type of entity that they point to

          \item To declare a pointer, use * preceding the variables name, ex: \code{int *x;}

        \end{itemize}
        
      \item To set a pointer to a variable's address, use \& before the variable, as in \code{x= \&y;}

        \begin{itemize}

          \item \& means ``return the memory address of''

          \item In this example, \code{x} will now point to \code{y}; that is, \code{x} stores \code{y}'s address

        \end{itemize}

      \item If you access \code{x}, you merely get the address

      \item To get the value that \code{x} points to, use *, as in \code{*x}

        \begin{itemize}

          \item \code{*x = *x + 1;} will add one to \code{y}

        \end{itemize}

      \item * is known as the indirection (or dereferencing) operator because it requires a second address

    \end{itemize}

\end{itemize}

\end{document}


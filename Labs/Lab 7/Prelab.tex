%%%%%%%%%%%%%%%%%%%%%%%%%%%%%%%%%%%%%%%%%%%%%%%%%%%%%%%%%%%%%%%%%%%%%%%%%%%%%%%%%%%%%%%%%%%%%%%%%%%%%%%%%%%%%%%%%%%%%%%%%%%%%%%%%%%%%%%%%%%%%%%%%%%%%%%%%%%%%%%%%%%
% Written By Michael Brodskiy
% Class: Embedded Design: Enabling Robotics
% Professor: S. Shazli
%%%%%%%%%%%%%%%%%%%%%%%%%%%%%%%%%%%%%%%%%%%%%%%%%%%%%%%%%%%%%%%%%%%%%%%%%%%%%%%%%%%%%%%%%%%%%%%%%%%%%%%%%%%%%%%%%%%%%%%%%%%%%%%%%%%%%%%%%%%%%%%%%%%%%%%%%%%%%%%%%%%

\include{Includes.tex}

\title{Lab 7 Pre-Lab Submission}
\date{\today}
\author{Michael Brodskiy\\ \small Professor: S. Shazli}

\begin{document}

\maketitle

\begin{figure}[H]
  \centering
  \includegraphics[width=.7\textwidth]{Figures/gdb.png}
  \caption{\texttt{gdb} output}
  \label{fig:1}
\end{figure}

The \texttt{gdb} commands may be explained as follows:

\begin{itemize}

  \item \texttt{file person} selects the binary called person as the file for analysis

  \item \texttt{start} begins analysis of ``person''

  \item \texttt{next} moves to the next point of interest

  \item \texttt{step} enters the function at the current line

  \item \texttt{print} prints the known information for a certain, specified value

\end{itemize}

\lstinputlisting[
    caption=Menu Printing Code, % Caption above the listing
    label=lst:L1, % Label for referencing this listing
    language=C++, % Use C++ functions/syntax highlighting
    frame=single, % Frame around the code listing
    showstringspaces=false, % Don't put marks in string spaces
    numbers=left, % Line numbers on left
    numberstyle=\tiny, % Line numbers styling
    backgroundcolor=\color{black!5}, % Set background color
    keywordstyle=\color{magenta!80}, % Set keyword color
    commentstyle=\color{blue!80}, % Set comment color
    stringstyle=\color{green!80}, % Set string color
    breaklines=true
  ]{Code/PreLab.cpp}

\begin{figure}[H]
  \centering
  \includegraphics[width=.9\textwidth]{Figures/output.png}
  \caption{Sample menu output}
  \label{fig:2}
\end{figure}

\end{document}


\documentclass[
	letterpaper, % Paper size, specify a4paper (A4) or letterpaper (US letter)
	10pt, % Default font size, specify 10pt, 11pt or 12pt
]{CSUniSchoolLabReport}

%----------------------------------------------------------------------------------------
%	REPORT INFORMATION
%----------------------------------------------------------------------------------------

\title{Memory-Mapped I/O and Object-Oriented Programming\\ Embedded Design: Enabling Robotics \\ EECE2160} % Report title

\author{Michael \textsc{Brodskiy}\\ \small \href{mailto:Brodskiy.M@Northeastern.edu}{Brodskiy.M@Northeastern.edu}}

\date{April 6, 2023} % Date of the report

%----------------------------------------------------------------------------------------


\begin{document}

\maketitle % Insert the title, author and date using the information specified above

\begin{center}
	\begin{tabular}{l r}
		Date Performed: & March 30, 2023 \\ % Date the experiment was performed
        Partner: & Dylan \textsc{Powers} \\ % Partner names
		Instructor: & Professor \textsc{Shazli} % Instructor/supervisor
	\end{tabular}
\end{center}

\newpage

\begin{abstract}

  This laboratory experiment, intended to introduce memory mapping and object-oriented programming concepts in C++, involved reading and writing to the DE1-SoC FPGA board. Starting with some pre-made register read and write code, conversions between binary and decimal number systems were used to infer the status of switches, LEDs, and push-buttons on the board. Functions were then written to interface with the board; these functions were then converted to an object-oriented program, with two classes: \texttt{DE1SoCfpga} and \texttt{LEDControl}.

\end{abstract}

\begin{flushleft}

  \textsc{Keywords:} \underline{memory mapping}, \underline{object-oriented}, \underline{read and write}, \underline{DE1-SoC}, \underline{binary}, \underline{decimal}, \underline{function}, \underline{class}

\end{flushleft}

\newpage

\section{Equipment}

\hspace{.5 in} Available equipment included:\\

\begin{itemize}

  \item DE1-SoC board

  \item DE1-SoC Power Cable

  \item USB-A to USB-B Cable

  \item Computer

  \item MobaXTerm SSH Terminal

  \item USB-to-ethernet Adapter

\end{itemize}

\section{Introduction}

Memory-mapping is a technique that accesses the virtual file representing I/O devices and maps a set of control flags into memory locations which enables a user to modify or read the state of the device. In Lab 8, the goal was to introduce the concept of memory-mapped I/O to access devices available on the DE1-SoC, including the LEDs, the switches, and the push buttons. To accomplish this, the \texttt{/dev/mem} file was the accessed file that was used to map physical addresses to virtual addresses. Then with the mapped relations between the physical and virtual addresses, programs were written to control the LEDs, switches, and push buttons using pointers.

\section{Discussion \& Analysis} 

\subsection{Pre-lab}

  \item Existing Functions:

    \begin{enumerate}

      \item \texttt{char *Initialize(int *fd)}

        First and foremost, \texttt{Initialize()} points the \texttt{fd} pointer at the location of a physical device in memory, and gives it read, write, and synchronization access. Next, an \texttt{if} statement checks for the possibility of errors; that is, if the pointer \texttt{fd} is equal to -1, then an error is printed, and the program exits with exit code 1 (error). The physical device is then mapped to a virtual device, and given a virtual memory address, using the \texttt{mmap()} function. Another \texttt{if} statement then checks whether the memory mapping was successful or not; if yes, the virtual memory location of the device is returned. Otherwise, an error is printed, the \texttt{fd} pointer connection to memory is closed, and exit code 1 is returned.

      \item \texttt{void Finalize(char *pBase, int fd)}

        The \texttt{Finalize()} function checks whether the device attached to the memory address pointer \texttt{pBase} is successfully unmapped from memory. If successful, it closes the connection to the device using address \texttt{fd}. Otherwise, an error is printed, and exit code 1 is returned.

      \item \texttt{int RegisterRead(char *pBase, unsigned int reg\_offset)}

        By combining the base address pointer \texttt{pBase} and the device mapping offset value (\texttt{reg\_offset}), the value that is read from the device at the memory address \texttt{pBase} is returned by \texttt{RegisterRead()}.

      \item \texttt{void RegisterWrite(char *pBase, unsigned int reg\_offset, int value)}

        Similar to \texttt{RegisterRead()}, \texttt{RegisterWrite} combines the base address pointer \texttt{pBase} and the device mapping offset, \texttt{reg\_offset}, to find the device, and then assigns a specified value to that address.

    \end{enumerate}

  \item Writing a switch read function:

   \lstinputlisting[
    caption=Switch Reading Code, % Caption above the listing
    label=lst:L1, % Label for referencing this listing
    language=C++, % Use C++ functions/syntax highlighting
    frame=single, % Frame around the code listing
    showstringspaces=false, % Don't put marks in string spaces
    numbers=left, % Line numbers on left
    numberstyle=\tiny, % Line numbers styling
    backgroundcolor=\color{black!5}, % Set background color
    keywordstyle=\color{magenta!80}, % Set keyword color
    commentstyle=\color{blue!80}, % Set comment color
    stringstyle=\color{green!80}, % Set string color
    breaklines=true
  ]{Code/partb.cpp}

\item Writing a switch write function:

 \lstinputlisting[
    caption=Switch Writing Code, % Caption above the listing
    label=lst:L2, % Label for referencing this listing
    language=C++, % Use C++ functions/syntax highlighting
    frame=single, % Frame around the code listing
    showstringspaces=false, % Don't put marks in string spaces
    numbers=left, % Line numbers on left
    numberstyle=\tiny, % Line numbers styling
    backgroundcolor=\color{black!5}, % Set background color
    keywordstyle=\color{magenta!80}, % Set keyword color
    commentstyle=\color{blue!80}, % Set comment color
    stringstyle=\color{green!80}, % Set string color
    breaklines=true
  ]{Code/partc.cpp}


\subsection{Assignment 1}

\subsection{Assignment 2}

\subsection{Assignment 3}

\subsection{Assignment 4}

\subsection{Assignment 5}

\section{Conclusion}

Overall, this lab was an effective introduction to the concept of C++ object-oriented programming. By first having the user create functions to interface with the board, and then having the user create a class, it was easier to grasp the idea of a class. The conversion of the program from procedural to object-oriented was, in this manner, facilitated.

\section{Appendix}

\lstinputlisting[
    caption=Object-Oriented Source Code, % Caption above the listing
    label=lst:L6, % Label for referencing this listing
    language=C++, % Use C++ functions/syntax highlighting
    frame=single, % Frame around the code listing
    showstringspaces=false, % Don't put marks in string spaces
    numbers=left, % Line numbers on left
    numberstyle=\tiny, % Line numbers styling
    backgroundcolor=\color{black!5}, % Set background color
    keywordstyle=\color{magenta!80}, % Set keyword color
    commentstyle=\color{blue!80}, % Set comment color
    stringstyle=\color{green!80}, % Set string color
    breaklines=true
  ]{Code/PushButtonClass.cpp}

\end{document}

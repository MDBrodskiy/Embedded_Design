%%%%%%%%%%%%%%%%%%%%%%%%%%%%%%%%%%%%%%%%%%%%%%%%%%%%%%%%%%%%%%%%%%%%%%%%%%%%%%%%%%%%%%%%%%%%%%%%%%%%%%%%%%%%%%%%%%%%%%%%%%%%%%%%%%%%%%%%%%%%%%%%%%%%%%%%%%%%%%%%%%%
% Written By Michael Brodskiy
% Class: Embedded Design: Enabling Robotics
% Professor: S. Shazli
%%%%%%%%%%%%%%%%%%%%%%%%%%%%%%%%%%%%%%%%%%%%%%%%%%%%%%%%%%%%%%%%%%%%%%%%%%%%%%%%%%%%%%%%%%%%%%%%%%%%%%%%%%%%%%%%%%%%%%%%%%%%%%%%%%%%%%%%%%%%%%%%%%%%%%%%%%%%%%%%%%%

\documentclass[12pt]{article} 
\usepackage{alphalph}
\usepackage[utf8]{inputenc}
\usepackage[russian,english]{babel}
\usepackage{titling}
\usepackage{amsmath}
\usepackage{graphicx}
\usepackage{enumitem}
\usepackage{amssymb}
\usepackage[super]{nth}
\usepackage{everysel}
\usepackage{ragged2e}
\usepackage{geometry}
\usepackage{multicol}
\usepackage{fancyhdr}
\usepackage{cancel}
\usepackage{siunitx}
\usepackage{physics}
\usepackage{lastpage}
\usepackage{tikz}
\usepackage{mathdots}
\usepackage{yhmath}
\usepackage{cancel}
\usepackage{color}
\usepackage{array}
\usepackage{multirow}
\usepackage{gensymb}
\usepackage{tabularx}
\usepackage{extarrows}
\usepackage{booktabs}
\usetikzlibrary{fadings}
\usetikzlibrary{patterns}
\usetikzlibrary{shadows.blur}
\usetikzlibrary{shapes}

\geometry{top=1.0in,bottom=1.0in,left=1.0in,right=1.0in}
\newcommand{\subtitle}[1]{%
  \posttitle{%
    \par\end{center}
    \begin{center}\large#1\end{center}
    \vskip0.5em}%

}
\usepackage{hyperref}
\hypersetup{
colorlinks=true,
linkcolor=blue,
filecolor=magenta,      
urlcolor=blue,
citecolor=blue,
}

\pagestyle{fancy}

\lfoot[\vspace{-15pt} \hline]{\vspace{-15pt} \hline}
\rfoot[\vspace{-15pt} \hline]{\vspace{-15pt} \hline}
\cfoot[\thepage]{\thepage}
\chead[\textsc{Embedded Systems}]{\textsc{Embedded Systems}}
\lhead[\textsc{EECE2160, CRN: 32014}]{\textsc{EECE2160, CRN: 32014}}
\rhead[\textsc{Page \thepage \hspace{1pt} of \pageref{LastPage}}]{\textsc{Page \thepage \hspace{1pt} of \pageref{LastPage}}}



\title{Lab 8 Pre-Lab Submission}
\date{\today}
\author{Michael Brodskiy\\ \small Professor: S. Shazli}

\begin{document}

\maketitle

\begin{enumerate}

  \item Existing Functions:

    \begin{enumerate}

      \item \texttt{char *Initialize(int *fd)}

        First and foremost, \texttt{Initialize()} points the \texttt{fd} pointer at the location of a physical device in memory, and gives it read, write, and synchronization access. Next, an \texttt{if} statement checks for the possibility of errors; that is, if the pointer \texttt{fd} is equal to -1, then an error is printed, and the program exits with exit code 1 (error). The physical device is then mapped to a virtual device, and given a virtual memory address, using the \texttt{mmap()} function. Another \texttt{if} statement then checks whether the memory mapping was successful or not; if yes, the virtual memory location of the device is returned. Otherwise, an error is printed, the \texttt{fd} pointer connection to memory is closed, and exit code 1 is returned.

      \item \texttt{void Finalize(char *pBase, int fd)}

        The \texttt{Finalize()} function checks whether the device attached to the memory address pointer \texttt{pBase} is successfully unmapped from memory. If successful, it closes the connection to the device using address \texttt{fd}. Otherwise, an error is printed, and exit code 1 is returned.

      \item \texttt{int RegisterRead(char *pBase, unsigned int reg\_offset)}

        By combining the base address pointer \texttt{pBase} and the device mapping offset value (\texttt{reg\_offset}), the value that is read from the device at the memory address \texttt{pBase} is returned by \texttt{RegisterRead()}.

      \item \texttt{void RegisterWrite(char *pBase, unsigned int reg\_offset, int value)}

        Similar to \texttt{RegisterRead()}, \texttt{RegisterWrite} combines the base address pointer \texttt{pBase} and the device mapping offset, \texttt{reg\_offset}, to find the device, and then assigns a specified value to that address.

    \end{enumerate}

  \item Writing a switch read function:

   \lstinputlisting[
    caption=Switch Reading Code, % Caption above the listing
    label=lst:L1, % Label for referencing this listing
    language=C++, % Use C++ functions/syntax highlighting
    frame=single, % Frame around the code listing
    showstringspaces=false, % Don't put marks in string spaces
    numbers=left, % Line numbers on left
    numberstyle=\tiny, % Line numbers styling
    backgroundcolor=\color{black!5}, % Set background color
    keywordstyle=\color{magenta!80}, % Set keyword color
    commentstyle=\color{blue!80}, % Set comment color
    stringstyle=\color{green!80}, % Set string color
    breaklines=true
  ]{Code/partb.cpp}

\item Writing a switch write function:

 \lstinputlisting[
    caption=Switch Writing Code, % Caption above the listing
    label=lst:L2, % Label for referencing this listing
    language=C++, % Use C++ functions/syntax highlighting
    frame=single, % Frame around the code listing
    showstringspaces=false, % Don't put marks in string spaces
    numbers=left, % Line numbers on left
    numberstyle=\tiny, % Line numbers styling
    backgroundcolor=\color{black!5}, % Set background color
    keywordstyle=\color{magenta!80}, % Set keyword color
    commentstyle=\color{blue!80}, % Set comment color
    stringstyle=\color{green!80}, % Set string color
    breaklines=true
  ]{Code/partc.cpp}

\end{enumerate}

\end{document}


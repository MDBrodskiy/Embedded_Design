\documentclass[
	letterpaper, % Paper size, specify a4paper (A4) or letterpaper (US letter)
	10pt, % Default font size, specify 10pt, 11pt or 12pt
]{CSUniSchoolLabReport}

%----------------------------------------------------------------------------------------
%	REPORT INFORMATION
%----------------------------------------------------------------------------------------

\title{Controlling Seven Segment Displays Using Object-Oriented Programming\\ Embedded Design: Enabling Robotics \\ EECE2160} % Report title

\author{Michael \textsc{Brodskiy}\\ \small \href{mailto:Brodskiy.M@Northeastern.edu}{Brodskiy.M@Northeastern.edu}}

\date{April 19, 2023} % Date of the report

%----------------------------------------------------------------------------------------


\begin{document}

\maketitle % Insert the title, author and date using the information specified above

\begin{center}
	\begin{tabular}{l r}
		Date Performed: & April 12, 2023 \\ % Date the experiment was performed
        Partner: & Dylan \textsc{Powers} \\ % Partner names
		Instructor: & Professor \textsc{Shazli} % Instructor/supervisor
	\end{tabular}
\end{center}

\newpage

\begin{abstract}

  This laboratory experiment was intended to be a conclusion to the course, which integrates all concepts covered, including, but not limited to, bits and hexadecimal, digital logic, object-oriented C++, and headers and makefiles. By integrating all of these concepts together, with minimal assistance, the course leaves us with proficient knowledge of them. As a result of the lab, three header files, \texttt{DE1SoCfpga.h}, \texttt{LEDControl.h}, and \texttt{SevenSegment.h}, and their respective \texttt{.cpp} files were created, in addition to a \texttt{main.cpp} file containing code to interact with headers, and a makefile to compile everything together.

\end{abstract}

\begin{flushleft}

  \textsc{Keywords:} \underline{bits}, \underline{hexadecimal}, \underline{digital logic}, \underline{object-oriented}, \underline{headers}, \underline{makefiles}, \underline{\texttt{DE1SoCfpga}}, \underline{\texttt{LEDControl}}, \underline{\texttt{SevenSegment}} 

\end{flushleft}

\newpage

\section{Equipment}

\hspace{.5 in} Available equipment included:\\

\begin{itemize}

  \item DE1-SoC board

  \item DE1-SoC Power Cable

  \item USB-A to USB-B Cable

  \item Computer

  \item MobaXTerm SSH Terminal

  \item USB-to-ethernet Adapter

  \item \text{gcc} compiler

\end{itemize}

\section{Introduction}

This lab project has the primary goal of controlling the 7-segment displays using object-oriented programming in C++. In the project, the 7-segment displays on the DE1-SoC board were utilized to display characters, decimal values and hexadecimal values being controlled by two parallel ports. The first port controls the displays (\texttt{HEX3}, \texttt{HEX2}, \texttt{HEX1}, \& \texttt{HEX0}), and the second parallel port controls the last two displays (\texttt{HEX5} \& \texttt{HEX4}). Data can be written into these two ports (registers) and read back by using word operations. 

\section{Discussion \& Analysis} 

\subsection{Assignment 1}

The purpose of assignment one was simply logic-based. It was necessary to consider the 7-bit logic behind seven-segment displays, and generate a table representing a hexadecimal number or letter in decimal, binary, and hexadecimal. The table is shown below.

   \begin{center}
      \begin{tabular}[h!]{| c | c | c | c | c | c | c | c | c | c |}
        \hline
        \# & 6 & 5 & 4 & 3 & 2 & 1 & 0 & Decimal & Hex \\
        \hline
        0 & 0 & 1 & 1 & 1 & 1 & 1 & 1 & 63 & 0x3F\\
        \hline
        1 & 0 & 0 & 0 & 0 & 1 & 1 & 0 & 6 & 0x6\\
        \hline
        2 & 1 & 0 & 1 & 1 & 0 & 1 & 1 & 91 & 0x5B\\
        \hline
        3 & 1 & 0 & 0 & 1 & 1 & 1 & 1 & 79 & 0x4F\\
        \hline
        4 & 1 & 1 & 0 & 0 & 1 & 1 & 0 & 102 & 0x66\\
        \hline
        5 & 1 & 1 & 0 & 1 & 1 & 0 & 1 & 109 & 0x6D\\
        \hline
        6 & 1 & 1 & 1 & 1 & 1 & 0 & 1 & 125 & 0x7D\\
        \hline
        7 & 0 & 0 & 0 & 0 & 1 & 1 & 1 & 7 & 0x7\\
        \hline
        8 & 1 & 1 & 1 & 1 & 1 & 1 & 1 & 127 & 0x7F\\
        \hline
        9 & 1 & 1 & 0 & 1 & 1 & 1 & 1 & 111 & 0x6F\\
        \hline
        A & 1 & 1 & 1 & 0 & 1 & 1 & 1 & 119 & 0x77\\
        \hline
        b & 1 & 1 & 1 & 1 & 1 & 0 & 0 & 124 & 0x7C\\
        \hline
        C & 0 & 1 & 1 & 1 & 0 & 0 & 1 & 57 & 0x39\\
        \hline
        d & 1 & 0 & 1 & 1 & 1 & 1 & 0 & 94 & 0x5E\\
        \hline
        e & 1 & 1 & 1 & 1 & 0 & 0 & 1 & 121 & 0x79\\
        \hline
        f & 1 & 1 & 1 & 0 & 0 & 0 & 1 & 113 & 0x71\\
        \hline
      \end{tabular}
    \end{center}

\subsection{Assignment 2}

In Part 2 of this project, the \texttt{DE1SoCfpga} class from the previous lab was converted into an independent class with its own header file and source file. The two files that were created are shown below in Listing \ref{lst:L1} and \ref{lst:L2}. The independent class \texttt{DE1SoCfpga} served as the base class for initializing memory and controlling register access.

\lstinputlisting[
    caption=\texttt{DE1SoCfpga} Class Declaration in Header File \texttt{DE1SoCfpga.h}, % Caption above the listing
    label=lst:L1, % Label for referencing this listing
    language=C++, % Use C++ functions/syntax highlighting
    frame=single, % Frame around the code listing
    showstringspaces=false, % Don't put marks in string spaces
    numbers=left, % Line numbers on left
    numberstyle=\tiny, % Line numbers styling
    backgroundcolor=\color{black!5}, % Set background color
    keywordstyle=\color{magenta!80}, % Set keyword color
    commentstyle=\color{blue!80}, % Set comment color
    stringstyle=\color{green!80}, % Set string color
    breaklines=true,
  ]{project/sevenLED/DE1SoCfpga.h}

\lstinputlisting[
    caption=\texttt{DE1SoCfpga} Implementation file \texttt{DE1SoCfpga.cpp}, % Caption above the listing
    label=lst:L2, % Label for referencing this listing
    language=C++, % Use C++ functions/syntax highlighting
    frame=single, % Frame around the code listing
    showstringspaces=false, % Don't put marks in string spaces
    numbers=left, % Line numbers on left
    numberstyle=\tiny, % Line numbers styling
    backgroundcolor=\color{black!5}, % Set background color
    keywordstyle=\color{magenta!80}, % Set keyword color
    commentstyle=\color{blue!80}, % Set comment color
    stringstyle=\color{green!80}, % Set string color
    breaklines=true,
  ]{project/sevenLED/DE1SoCfpga.cpp}

  When looking at the implementation file, \texttt{DE1SoCfpga.cpp}, the functions for the previous lab were implemented from the previous lab. The constructor \texttt{DE1SoCfpga()} was included to initialize the memory-mapped I/O, the destructor \texttt{\char`~DE1SoCfpga()} was included to finalize the memory-mapped I/O, the function \texttt{RegisterWrite(offset, value)} was a included to write a value to into a register given the offset, and the function \texttt{RegisterRead(offset)} was included to return the value read from a register given its offset.

\subsection{Assignment 3}

In Part 3 of the lab project, a new class called \texttt{SevenSegment} was created. This class had a declaration file or header file called \texttt{SevenSegment.h} and an implementation file or source file called \texttt{SevenSegment.cpp} both of which were in the same sevenLED directory. The two files are shown below in Listing \ref{lst:L3} and \ref{lst:L4}.

\lstinputlisting[
    caption=\texttt{SevenSegment} class declaration in header file \texttt{SevenSegment.h}, % Caption above the listing
    label=lst:L3, % Label for referencing this listing
    language=C++, % Use C++ functions/syntax highlighting
    frame=single, % Frame around the code listing
    showstringspaces=false, % Don't put marks in string spaces
    numbers=left, % Line numbers on left
    numberstyle=\tiny, % Line numbers styling
    backgroundcolor=\color{black!5}, % Set background color
    keywordstyle=\color{magenta!80}, % Set keyword color
    commentstyle=\color{blue!80}, % Set comment color
    stringstyle=\color{green!80}, % Set string color
    breaklines=true,
  ]{project/sevenLED/SevenSegment.h}

\lstinputlisting[
    caption=\texttt{SevenSegment} source file \texttt{SevenSegment.cpp}, % Caption above the listing
    label=lst:L4, % Label for referencing this listing
    language=C++, % Use C++ functions/syntax highlighting
    frame=single, % Frame around the code listing
    showstringspaces=false, % Don't put marks in string spaces
    numbers=left, % Line numbers on left
    numberstyle=\tiny, % Line numbers styling
    backgroundcolor=\color{black!5}, % Set background color
    keywordstyle=\color{magenta!80}, % Set keyword color
    commentstyle=\color{blue!80}, % Set comment color
    stringstyle=\color{green!80}, % Set string color
    breaklines=true,
  ]{project/sevenLED/SevenSegment.cpp}


  In these two files, there was important functionality included to achieve the goal of controlling the 7-segment displays on the DE1-SoC board. In the \texttt{SevenSegment.h} file two new private, unsigned int data members \texttt{reg0\_hexValue} and \texttt{reg1\_hexValue} were created with the ability to update every time a new value was written to the corresponding register as well as a global array with 16 elements with each element representing the display segments to be turned ON or OFF when writing to a 7-segment display. Here in the \texttt{SevenSegment.h} file is also where the class \texttt{SevenSegment} inherits the class \texttt{DE1SoCfpga} and uses the functions \texttt{RegisterWrite()} and \texttt{RegisterRead()} as needed. \\

  Then in the \texttt{SevenSegment.cpp} file, the constructor \texttt{SevenSegment()} was included as well as the destructor \texttt{\char`~SevenSegment()} which called a new function \texttt{Hex\_ClearAll()}. \texttt{Hex\_ClearAll()} is a function included in the \texttt{SevenSegment.cpp} source file that clears all the 7-segment displays. This was done by writing the value zero to the two data registers for the HEX displays with the \texttt{RegisterWrite()} function. Along with the function \texttt{Hex\_ClearAll()}, the functions \texttt{Hex\_ClearSpecific(int index)}, \texttt{Hex\_WriteSpecific(int display\_id, int value)} and \texttt{Hex\_WriteNumber(int number)} were created and included in the \texttt{SevenSegment.cpp} file. Starting with the function \texttt{Hex\_ClearSpecific()} this function turns off a specific 7-segment display specified by an index (0 to 5) using a bit mask and bit shifting. Next with the function \texttt{Hex\_WriteSpecific()}, this public function enables the ability to write a hexadecimal digit specified by a decimal value (0-15) to a specified display indicated by a \texttt{display\_id} (0-5). This function was created using an if else statement and the concepts of bit masking and bit shifting. Lastly the function \texttt{Hex\_WriteNumber(int number)} writes a positive or negative number on the 7-segment displays as shown in the \texttt{SevenSegment} source file \texttt{SevenSegment.cpp} shown above. \\

  All functions described above were verified using the \texttt{main.cpp} file shown below in Listing \ref{lst:L5} to ensure correct behavior.

\lstinputlisting[
    caption=\texttt{main.cpp} file for verifying behavior of functions in \texttt{SevenSegment.cpp}, % Caption above the listing
    label=lst:L5, % Label for referencing this listing
    language=C++, % Use C++ functions/syntax highlighting
    frame=single, % Frame around the code listing
    showstringspaces=false, % Don't put marks in string spaces
    numbers=left, % Line numbers on left
    numberstyle=\tiny, % Line numbers styling
    backgroundcolor=\color{black!5}, % Set background color
    keywordstyle=\color{magenta!80}, % Set keyword color
    commentstyle=\color{blue!80}, % Set comment color
    stringstyle=\color{green!80}, % Set string color
    breaklines=true,
  ]{project/seven/main.cpp}

  In order to complete the verification process, a Makefile had to be created to compile the programs \texttt{DE1SoCfpga.h}, \texttt{DE1SoCfpga.cpp}, \texttt{SevenSegment.h}, \texttt{SevenSegment.cpp}, and \texttt{main.cpp}. 

\lstinputlisting[
    caption=, % Caption above the listing
    label=lst:L6, % Label for referencing this listing
    frame=single, % Frame around the code listing
    showstringspaces=false, % Don't put marks in string spaces
    numbers=left, % Line numbers on left
    numberstyle=\tiny, % Line numbers styling
    backgroundcolor=\color{black!5}, % Set background color
    keywordstyle=\color{magenta!80}, % Set keyword color
    commentstyle=\color{blue!80}, % Set comment color
    stringstyle=\color{green!80}, % Set string color
    breaklines=true,
  ]{project/seven/makefile}

Results of the verification test are shown in a video that can be accessed by the first QR code below.

\begin{figure}[h!]
  \centering
  \includegraphics[width=.55\textwidth]{Figures/Video1.png}
  \caption{Verification video of \texttt{SevenSegment.cpp} functions}
  \label{fig:1}
\end{figure}

\subsection{Assignment 4}

The objective of Part 4 for the project was to combine a program written in a previous lab with the program written in parts 2 and 3 of this project so that the value controlled by push buttons is displayed on the 7-segment displays in addition to being displayed on the LEDs. This meant using the class \texttt{DE1SoCfpga} as a base class with both classes \texttt{LEDControl} and \texttt{SevenSegment} inheriting this class. To start, the \texttt{LEDControl} class was converted into an independent class with its own header and source file shown in Listing \ref{lst:L7} and \ref{lst:L8} respectively.

\lstinputlisting[
    caption=\texttt{LEDControl} class declaration in header file \texttt{LEDControl.h}, % Caption above the listing
    label=lst:L7, % Label for referencing this listing
    language=C++, % Use C++ functions/syntax highlighting
    frame=single, % Frame around the code listing
    showstringspaces=false, % Don't put marks in string spaces
    numbers=left, % Line numbers on left
    numberstyle=\tiny, % Line numbers styling
    backgroundcolor=\color{black!5}, % Set background color
    keywordstyle=\color{magenta!80}, % Set keyword color
    commentstyle=\color{blue!80}, % Set comment color
    stringstyle=\color{green!80}, % Set string color
    breaklines=true
  ]{project/sevenLED/LEDControl.h}

\lstinputlisting[
    caption=\texttt{LEDControl} source file \texttt{LEDControl.cpp}, % Caption above the listing
    label=lst:L8, % Label for referencing this listing
    language=C++, % Use C++ functions/syntax highlighting
    frame=single, % Frame around the code listing
    showstringspaces=false, % Don't put marks in string spaces
    numbers=left, % Line numbers on left
    numberstyle=\tiny, % Line numbers styling
    backgroundcolor=\color{black!5}, % Set background color
    keywordstyle=\color{magenta!80}, % Set keyword color
    commentstyle=\color{blue!80}, % Set comment color
    stringstyle=\color{green!80}, % Set string color
    breaklines=true
  ]{project/sevenLED/LEDControl.cpp}


  In the \texttt{LEDControl.cpp} source file, the constructor \texttt{LEDControl()} and the destructor \texttt{\char'~LEDControl} were created where the destructor also displayed a message ``Closing LEDs, Switches, \& Buttons\ldots''. As for the functions \texttt{Write1Led()}, \texttt{WriteAllLeds()}, \texttt{Read1Switch()} and \texttt{ReadAllSwitches()} were included from the previous lab where they were tested and verified for correct behavior.  

  Then, in order to test the functions, a \texttt{main.cpp} file was created with the functionality of controlling the LEDs and Hex displays with the switches and the push buttons on the DE1-SoC board. The \texttt{main.cpp} file can be seen below.

\lstinputlisting[
    caption=\texttt{Main.cpp} for Part 4, % Caption above the listing
    label=lst:L9, % Label for referencing this listing
    language=C++, % Use C++ functions/syntax highlighting
    frame=single, % Frame around the code listing
    showstringspaces=false, % Don't put marks in string spaces
    numbers=left, % Line numbers on left
    numberstyle=\tiny, % Line numbers styling
    backgroundcolor=\color{black!5}, % Set background color
    keywordstyle=\color{magenta!80}, % Set keyword color
    commentstyle=\color{blue!80}, % Set comment color
    stringstyle=\color{green!80}, % Set string color
    breaklines=true
  ]{project/sevenLED/Main.cpp}

After this was completed, a Makefile was created to compile the programs, including a clean rule accessible through the command make clean in order to get rid of all generated binary files.

\lstinputlisting[
    caption=Makefile for Part 4, % Caption above the listing
    label=lst:L10, % Label for referencing this listing
    frame=single, % Frame around the code listing
    showstringspaces=false, % Don't put marks in string spaces
    numbers=left, % Line numbers on left
    numberstyle=\tiny, % Line numbers styling
    backgroundcolor=\color{black!5}, % Set background color
    keywordstyle=\color{magenta!80}, % Set keyword color
    commentstyle=\color{blue!80}, % Set comment color
    stringstyle=\color{green!80}, % Set string color
    breaklines=true
  ]{project/sevenLED/makefile}

With the Makefile, the program was compiled and tested to verify the LEDs displayed the inputs binary value while the 7-segment displays displayed the same value in decimal, reflecting changes to the value when the push buttons are pressed. Below is a video demonstrating the working design. 

\begin{figure}[h!]
  \centering
  \includegraphics[width=.55\textwidth]{Figures/Video2.png}
  \caption{Verification Video}
  \label{fig:2}
\end{figure}

\section{Conclusion}

Overall, this laboratory project was an effective way to finish off the course. By having us draw from concepts learned throughout the entirety of the course, we were able to effectively work with a hardware device integrated with C++. As such, through the creation of new code, as well as integration of code from previous labs, DE1SoCfpga board interaction was converted to a fully object-oriented C++ program, encompassing all course concepts.

\end{document}
